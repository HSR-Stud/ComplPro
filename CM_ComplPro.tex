%%%%%%%%%%%%%%%%%%%%%%%%
% Document Information %
%%%%%%%%%%%%%%%%%%%%%%%%
\title{CM\_ComplPro}
\author{Lukas Leuenberger}

\documentclass[10pt,twoside,a4paper,fleqn]{scrartcl}

\usepackage{longtable}
\usepackage{hhline}
\usepackage{textcomp}
\usepackage{float}
\usepackage{paralist}
\usepackage{framed}

\include{header/zusammenfassung}

\usepackage{colortbl}
\definecolor{lightgrey}{rgb}{0.9,0.9,0.9}
\usepackage{tikz}
\usetikzlibrary{shapes.geometric, arrows, positioning}

\tikzstyle{decision} = [rectangle, minimum width=2.5cm, minimum height=1em, text centered, draw=black, fill=white]
\tikzstyle{event} = [circle, minimum width=1.7cm, minimum height=1em, text centered, draw=black, fill=white, text width=1.7cm]
\tikzstyle{answer} = [rectangle, minimum width=2.5cm, minimum height=1em, draw=gray!15, fill=gray!15, text width=2.5cm]
\tikzstyle{answerW} = [rectangle, minimum width=2.5cm, minimum height=1em, draw=white, fill=white, text width=2.5cm]
\tikzstyle{decisionW} = [rectangle, minimum width=2.5cm, minimum height=1em, text centered, draw=black, fill=white, text width=6cm]
\tikzstyle{arrow} = [thick,->,>=stealth]

% Setze etwas in Anführungs und Schlusszeichen
\newcommand{\aszeichen}[1]{,,#1``}
\newcommand{\integral}[4]{\int_{#1}^{#2} \! {#3} \, \mathrm{d}{#4}}
\newcommand{\differential}[2]{\frac{\mathrm{d}{(#1)}}{\mathrm{d}{#2}}}
\newenvironment{example}[1][Beispiel]{\begingroup\setlength{\OuterFrameSep}{0pt}\colorlet{shadecolor}{gray!15}\begin{snugshade*}\textbf{#1: }}{\end{snugshade*}\endgroup}

% Spacing von multicols
\setlength\multicolsep{2pt}

\raggedbottom
% No additions here if possible, only in header
\begin{document}

% Set language to English so that table captions etc. are in German.
\selectlanguage{ngerman}

\section{Einführung}
\subsection{Komplizierte vs. komplexe Systeme}
Ein System ist komplex/kompliziert, wenn sein Gesamtverhalten trotz vollständiger Information über die Einzelkomponenten schwierig zu verstehen ist – auf Grund ...
\begin{multicols}{2}
	\textbf{Komplex:} ... von Rück- und Nebenwirkungen, Verzögerungen und Nichtlinearitäten der Ursache-Wirkungsbeziehungen. \\ \ \\
	\textbf{Kompliziert:} ... der grossen Zahl von Einzelkomponenten. Ein kompliziertes System ist 'reduzibel', d.h. es kann durch Abstraktion in eine einfaches System überführt werden.
\end{multicols}

\subsection{Fahndungsbild eines komplexen Systemes}
\todo{Allenfalls entfernen, weiss nicht wie wichtig das wirklich ist}
\begin{multicols}{3}
	\begin{compactitem}
		\item Hohe Anzahl Elemente
		\item Nichtlineare Wechselwirkungen zwischen den Elementen
		\item Verzögerte Auswirkungen		
		\item Netzartige Struktur
		\item Negative und positive Rückkoppelungen
		\item Offen
		\item Universell
		\item Dynamisch
		\item Robust
		\item Kreativ und innovativ
		\item Unvorhersehbar
		\item Differenzierte Sensibilität
		\item Nicht kontrollierbar
	\end{compactitem}
\end{multicols}

\subsection{Komplexitätsmanagement}
Komplexitätsmanagement heisst,
\begin{compactitem}
	\item die zentralen Ursache-Wirkungszusammenhänge zu verstehen
	\item zu erkennen, welche Abhängigkeiten nebensächlich sind
	\item steuerbare Einflussgrössen („Stellhebel“) und nicht-steuerbare Einflussgrössen (``exogene Faktoren'') zu erkennen.
	\item was-wäre-wenn-Fragen beantworten zu können (Simulation).
\end{compactitem}
\section{Deskriptive Entscheidungstheorie}
\subsection{Aspekte des Entscheidens}
\begin{multicols}{2}
	\textbf{Rationales Entscheiden:} 
	\begin{compactitem}
		\item Entscheidungsprozess ist zielgerichtet und orientiert sich konsequent an Zielen
		\item Überlegungen basieren auf möglichst objektiven Informationen
		\item Entscheidungsprozess folgt systematischen Vorgehen und verwendet klare methodische Regeln
	\end{compactitem}

	\textbf{Intuitives Entscheiden:} 
	\begin{compactitem}
		\item Lassen sich nicht zwingend nachvollziehen
		\item Entscheidunglogik oftmals auch der entscheidenden Person unbekannt
		\item Keine quantitative Bewertung
	\end{compactitem} \ \\
\end{multicols}

\subsection{Mentales Modell}
Vereinfachte mentale Abbildung der subjektiv wahrgenommenen Aussenwelt mit dem Zweck,
\begin{compactitem}
	\item die durch die Sinnesorgane aufgenommene grosse Informationsmenge zu bewältigen.
	\item Entscheidungen schnell fällen zu können.
\end{compactitem}
Beispiele: Duschwasser einstelle, Strasse überqueren, Schuhe binden, ... \\
Diese mentalen Prozessmodelle sind unbewusst, mittels Erfahrung gewonnen und werden sofort angepasst. \\
Nutzen: Informationsverarbeitung, Effizienzsteigerung, Notwendig um komplexe Sachverhalte zu interpretieren

\subsubsection{Probleme}
\begin{compactenum}
	\item Aberglaube (Ticket-Automat im Bus)
	\item Vorurteile (Rosenhan-Experiment - Psychiatrie)
	\item Mentale Modelle können eine Eigendynamik entwickeln (selektive Suche nach Bestätigung)
	\item Konservativismus: Mentale Modelle entwickeln sich mit der	Erfahrung
	\item Scheinsicherheit: Wenn mentale Modelle nicht als solche wahrgenommen werden, werden sie nicht hinterfragt
\end{compactenum}
Mentale Modelle lassen sich nicht verifizieren!

\subsection{Kausalität vs. Korrelation}
\begin{multicols}{2}
	\textbf{Korrelation:}
	Zwei Grössen sind korreliert, wenn sich statistisch eine mathematische Abhängigkeit nachweisen lässt.
	\ \\ \\
	\textbf{Kausalität:}
	Zwischen zwei Grössen besteht ein kausaler Zusammenhang, wenn zwischen Ihnen eine Ursache-Wirkungsbeziehung besteht.
\end{multicols}

\subsection{Falsifikationismus}
Bei Eintreffen eines bestimmten experimentellen Befunds wird die Hypothese verworfen.\\
\textbf{Beispiel:} Alle Schwäne sind weiss (Hypothese). Noch so viele weisse Schwäne beweisen die Aussage nicht. Wissenschaftliche Arbeit: Suche nach einem allfälligen schwarzen Schwan.
\section{Normative Entscheidungstheorie}
\begin{example}[Beispiel für Kapitel 3.1 - 3.4]
	Angenommen ein Mineralölkonzern strebe nach Gewinnmaximierung und überprüfe die Eröffnung einer neuen Tankstelle. In Frage kommen zwei Standorte:
	\begin{multicols}{2}
		\begin{compactenum}
			\item Standort im Stadtzentrum (a1): 
			\begin{compactitem}
				\item Erwarteter Gewinn 125'000.- CHF 
				\item Erwarteter Umsatz 2'000'000.- CHF
			\end{compactitem}
			\item Standort am Stadtrand (a2): 
			\begin{compactitem}
				\item Erwarteter Gewinn 150'000.- CHF 
				\item Erwarteter Umsatz 1'800'000.- CHF
			\end{compactitem}
		\end{compactenum}
	\end{multicols}
\end{example}

\subsection{Entscheidung unter Unsicherheit}
\textbf{Hauptsächliche Beschäftigungsgebiet:} Unsicherheit und nicht der Zielkonflikt \\
\textbf{Umweltzustände:} Faktoren auf welche man selber aber keinen Einfluss hat, die aber die Entscheidung beeinflussen\\
Bei Entscheidungssituationen in denen eine bestimmte Entscheidung nicht mit einem einzigen möglichen Ereignis verbunden ist, wird von einer Entscheidung unter Unsicherheit gesprochen.

\subsection{Ergebnismatrix}
Darstellungsform für Entscheidungen unter Unsicherheit wobei die Handlungsalternativen ($a_i$) in den Zeilen und die verschiedenen Umweltzustände ($z_i$) in den Spalten dargestellt werden. 
\begin{example}
	Dem Mineralölkonzern ist bekannt, dass eine weiträumige Umgehungsstrasse geplant ist, die aber höchst umstritten ist.
	\begin{compactitem}
		\item Diese würden den Verkehr im Stadtzentrum nicht tangieren und auf CHF 125'000.- belassen.
		\item Am Stadtrand würde sich aber eine drastische Veränderung ergeben,	die den erwarteten Gewinn auf CHF 80'000.- reduzieren liesse.
	\end{compactitem}
	Welcher Standort wäre unter der Prämisse der Gewinnmaximierung nun zu präferieren? \\
	\begin{tabular}{|l|l|l|}
		\hline
		& $z_1$ (keine Umgehungsstrasse) & $z_2$ (Umgehungsstrasse) \\ \hline
		$a_1$ (Stadtzentrum) & 125'000.- = $e_{11}$ & 125'000.- = $e_{11}$ \\ \hline
		$a_2$ (Stadtrand) & 150'000.- = $e_{21}$ & 80'000.- = $e_{22}$ \\ \hline
	\end{tabular}
\end{example}

\subsubsection{Zustandsraum}
\begin{compactitem}
	\item Raum aller möglicher Entscheidungsergebnisse.
	\item Die Entscheidungsmatrix beschreibt die Entscheidungssituation nur dann hinreichend, wenn der Zustandsraum voll-ständig erfasst wird.
	\item In der Realität oft nicht gegeben.
\end{compactitem}

\subsection{Klassifikation}
\textbf{Entscheidung unter Risiko:} Falls den Umweltzuständen Eintrittswahrscheinlichkeiten zugeordnet werden können. \\
\textbf{Entscheidung unter Ungewissheit:} Falls den Umweltzuständen keine Wahrscheinlichkeiten zugeordnet werden können. \\
Wahrscheinlichkeiten müssen nicht mathematisch fundiert sein (objektiv), sondern können sich auch aus Sicht des Entscheiders ergeben (subjektiv).
\begin{example}
	Angenommen der Ölkonzern schätzt die Wahrscheinlichkeit für den Bau	der Umgehungsstrasse auf 30\%. \\
	\begin{tabular}{|l|l|l|}
		\hline
		Alternative & $z_1$ (keine Umgehungsstrasse) & $z_2$ (Umgehungsstrasse) \\ 
		& $p_1$ = 0.7 & $p_2$ = 0.3 \\ \hline
		$a_1$ (Stadtzentrum) & 125'000.- = $e_{11}$ & 125'000.- = $e_{11}$ \\ \hline
		$a_2$ (Stadtrand) & 150'000.- = $e_{21}$ & 80'000.- = $e_{22}$ \\ \hline
	\end{tabular}
\end{example}
\textbf{Entscheidung wegen rationalem Gegenspieler:} Umweltzustände werden durch einen rationalen Gegenspieler bestimmt (und nicht durch Zufall)., gehört zur Spieltheorie.
\begin{example}
	Angenommen zwei verschiedene Ölkonzerne würden jeweils überlegen, ob sie am Stadtrand oder im Stadtzentrum eine neue Tankstelle errichten. \\
	Ungünstig wäre sicherlich am gleichen Standort wie der Konkurrent zu bauen. Entsprechend wäre es wünschenswert zu wissen wo der
	Konkurrent baut. Am besten wäre glaubhaft machen zu können, dass man am lukrativsten Standort plant zu bauen. Dann würde der Konkurrent auf den weniger lukrativen ausweichen.
\end{example}

\subsection{Dominanz}
\begin{compactitem}
	\item Alternative ist dominant, wenn sie einer anderen Alternative auf jeden Fall vorzuziehen ist.
	\item Dominiert eine Alternative alle anderen, so ist diese zu präferieren.
	\item Häufig kann so eine Vorauswahl getroffen werden, indem alle dominierten Alternativen entfernt werden.
\end{compactitem}

\subsubsection{Absolute Dominanz}
Der schlechteste Ergebniswert der dominierenden Alternative ist besser als der beste Ergebniswert der
dominierten Alternative.
\begin{example} \\
	\begin{tabular}{|l|l|l|}
		\hline
		Alternative & $z_1$ (keine Umgehungsstrasse) & $z_2$ (Umgehungsstrasse) \\ \hline
		$a_1$ (Stadtzentrum) & 125'000.- & 90'000.- \\ \hline
		$a_2$ (Stadtrand) & 100'000.- & 80'000.- \\ \hline
	\end{tabular}\\ \ \\
	Keine absolute Dominanz, da der schlechteste Wert der Alternative $a_1$ schlechter als der beste der Alternative $a_2$ ist.
\end{example}

\subsubsection{Zustandsdominanz}
Eine Alternative ist in jedem Zustand (Umweltzustand) besser als (bzw. gleich gut wie) die andere Alternative.
\begin{example} 
	In Tabelle von \aszeichen{Absolute Dominanz} ist unabhängig vom Umweltzustand die Alternative $a_1$ immer besser. $a_1$ dominiert $a_2$.
\end{example}

\subsection{Ziele}
\textbf{Komplementäre Ziele:} Durch die Verfolgung eines Ziels wird auch das andere Ziel optimal erreicht. \\
\textbf{Neutrale Ziele:} Die Verfolgung eines Ziels beeinflusst das andere Ziel nicht. \\
\textbf{Konkurrierende Ziele:} Die Realisierung des einen Ziels beeinträchtigt die Erreichung des anderen Ziels.

\subsubsection{Effizienz}
Eine Alternative ist effizient, wenn es keine andere Alternative gibt, die bezüglich mindestens eines Zieles besser und bezüglich keines Ziels schlechter ist.
\begin{example} 	
	Ein Chemieunternehmen will sowohl den Absatz als auch den Gewinn maximieren. Es vermutet allerdings, dass die zukünftigen Gewinne durch die von der Bevölkerung wahrgenommene Umweltschädigung verursacht durch das Unternehmen beeinflusst werden. Entsprechend gilt es auch die Schäden zu minimieren. Das Unternehmen kann über den Preis den Absatz steuern. Gewinn und Schäden ergeben sich dann automatisch:\\
	\begin{tabular}{|l|l|l|l|l|}
		\hline
		\textbf{Alternative} & \textbf{Preis} & \textbf{Absatz} & \textbf{Gewinn} & \textbf{Umweltschäden} \\ \hline
		$a_1$ & 15 & 800 & 7000 & -4 \\ \hline
		$a_2$ & 20 & 600 & 7000 & -2 \\ \hline
		$a_3$ & 25 & 400 & 6000 & 0 \\ \hline
		$a_4$ & 30 & 200 & 4000 & 0 \\ \hline
	\end{tabular} \\ \ \\
	$a_4$ ist ineffizient, da diese nie besser ist als $a_3$, aber bezüglich Absatz und Gewinn schlechter als $a_3$.
\end{example}
Sinnvoll ist es, sich nur auf die effizienten Alternativen zu beschränken.

\subsubsection{Zieldominanz}
Wenn ein Ziel als das entscheidende erachtet wird, kann die Erfüllung aller anderen Ziele vernachlässigt werden.

\subsubsection{Lexikographische Ordnung}
Bei Verfahren der Lexikographischen Ordnung werden die Ziele sortiert und nur wenn die Ziele identisch sind, wird das nächste Ziel zur Sortierung hinzugezogen.
\begin{example}
	Es wird angenommen, dass für die Chemiefirma die Gewinnmaximierung das Hauptziel ist und die Minimierung der Umweltschäden das zweitwichtigste Ziel. \\
	\begin{tabular}{|l|l|l|l|l|}
		\hline
		\textbf{Alternative} & \textbf{Preis} & \textbf{Absatz} & \textbf{Gewinn} & \textbf{Umweltschäden} \\ \hline
		$a_2$ & 20 & 600 & 7000 & -2 \\ \hline
		$a_1$ & 15 & 800 & 7000 & -4 \\ \hline		
		$a_3$ & 25 & 400 & 6000 & 0 \\ \hline
	\end{tabular}
\end{example}
\textbf{Problem:} Selbst ein minimaler Unterschied beim Hauptziel hat eine grössere Bedeutung als die grösste Abweichung im Nebenziel.

\subsubsection{Nutzenfunktion}
Entscheidungsproblem mit mehreren Zielen kann gelöst werden, indem das Maximum einer Nutzenfunktion, die als Variable die
verschiedenen Ziele enthält, bestimmt wird. \\
Suche den maximalen Wert für $U(e_{i1}, ..., e_{ij})$

\subsubsection{Zielgewichtung}
Spezieller Fall der Nutzenfunktion, bei der die einzelnen Ziele linear in die Nutzenfunktion eingehen.\\
\textbf{Vorgehen:} 
\begin{compactenum}
	\item Gewichte den einzelnen Zielen zuordnen.
	\item Normieren der Werte und anschliessend gewichten.
	\item Ergebniswerte für alle Alternativen addieren.
	\item Gewählt wird Alternative mit höchstem Wert.
\end{compactenum}
\begin{example}
	Eine Firma möchte sowohl den Umsatz	wie auch den Gewinn maximieren. \\
	\begin{tabular}{|l|l|l|}
		\hline
		\textbf{Alternative} & \textbf{Umsatz} & \textbf{Gewinn} \\ \hline
		$a_1$ & 800'000 & 7000 \\ \hline
		$a_2$ & 600'000 & 8000 \\ \hline		
	\end{tabular} \\ \ \\
	Angenommen der Firma wäre nun die Gewinnmaximierung dreimal wichtiger als die Umsatzmaximierung würden die Gewichtungsfaktoren folgendermassen lauten: 
	\begin{compactitem}
		\item $a_1$ : 0.25 * 800'000 + 0.75 * 7000 = 205'250
		\item $a_2$ : 0.25 * 600'000 + 0.75 * 8000 = 156'000
	\end{compactitem}
\end{example}

\subsubsection{Normierung}
Eine vorgängige Normierung der Ergebnisse kann mittels folgender Formel erfolgen: \\
$\text{normierter Wert} = \frac{\text{zu normierender Wert} - \text{minimaler Wert}}{\text{maximaler Wert} - \text{minimaler Wert}}$
\begin{example}
	\begin{compactitem}
		\item $a_1$ : 0.25 * 1 + 0.75 * 1 = 0.25
		\item $a_2$ : 0.25 * 0 + 0.75 * 0 = 0.75
	\end{compactitem}
\end{example}

\subsection{Entscheidung unter Ungewissheit}
Situation mit mehreren Umweltzuständen, für die die Eintrittswahrscheinlichkeiten nicht bekannt sind.

\subsubsection{Maximin-Regel}
\textbf{Regel:} Man wähle die Handlungsalternative mit dem maximale Minimum. \\
\textbf{Intention:} Extrem risikoscheue Entscheidungsregel, da nur das schlechteste mögliche Ergebnis in die Beurteilung einfliesst.\\
\textbf{Kritik:} Realitätsfremd und es wird nur ein einziger Wert einer Alternative berücksichtigt.
\begin{example} \\
	\begin{tabular}{|l|l|l|l|l|l|}
		\hline
		& $z_1$ & $z_2$ & $z_3$ & $z_4$ & Minimum \\ \hline
		$a_1$ & 60 & 30 & 50 & 60 & \textbf{30} \\ \hline
		$a_2$ & 10 & 10 & 10 & 140 & 10 \\ \hline
		$a_3$ & -30 & 100 & 120 & 130 & -30 \\ \hline		
	\end{tabular}
\end{example}

\subsubsection{Maximax-Regel}
\textbf{Regel:} Man wähle die Handlungsalternative mit dem maximale Maximum. \\
\textbf{Intention:} Extrem risikofreudige (optimistische) Entscheidungsregel, da nur das bestmögliche Ergebnis in die Beurteilung einfliesst.\\
\textbf{Kritik:} Extrem realitätsfremd und es wird ebenfalls nur ein einziger Wert einer Alternative berücksichtigt.
\begin{example} \\
	\begin{tabular}{|l|l|l|l|l|l|}
		\hline
		& $z_1$ & $z_2$ & $z_3$ & $z_4$ & Maximum \\ \hline
		$a_1$ & 60 & 30 & 50 & 60 & 60 \\ \hline
		$a_2$ & 10 & 10 & 10 & 140 & \textbf{140} \\ \hline
		$a_3$ & -30 & 100 & 120 & 130 & 130 \\ \hline		
	\end{tabular}
\end{example}

\subsubsection{Hurwicz-Regel}
\textbf{Regel:} Kombination aus der Maximin- und der Maximax-Regel durch Einführung eines Optimismusparameters $\lambda$. \\
$\Phi(a_i) = \lambda * \max_j(e_{ij}) + (1 - \lambda) * \min_j(e_{ij})$ mit $0 \leq \lambda \leq 1$ \\
\textbf{Intention:} Entscheidungsregel für Entscheider, die weder absolut optimistisch noch absolut pessimistisch sind.\\
\textbf{Kritik:} In vielen Fällen realitätsfremd und es werden nur zwei Werte jeder Alternative berücksichtigt.
\begin{example} 
	$\lambda$ = 0.4 \\
	\begin{tabular}{|l|l|l|l|l|l|}
		\hline
		& $z_1$ & $z_2$ & $z_3$ & $z_4$ & $\Phi(a_i)$ \\ \hline
		$a_1$ & 60 & 30 & 50 & 60 & 0.4 * 60 + 0.6 * 30 = 42 \\ \hline
		$a_2$ & 10 & 10 & 10 & 140 & 0.4 * 140 + 0.6 * 10 = \textbf{62} \\ \hline
		$a_3$ & -30 & 100 & 120 & 130 & 0.4 * 130 + 0.6 * (-30) = 34 \\ \hline		
	\end{tabular}
\end{example}

\subsubsection{Savage-Niehans-Regel}
\textbf{Regel:} \aszeichen{Regel des kleinsten Bedauerns} \\
\textbf{Intention:} Die Regel versucht das maximal mögliche Bedauern zu minimieren.\\
\textbf{Vorgehen:} Aufstellen einer Matrix des Bedauerns, indem die Maxima der Umweltzustände ermittelt und dann pro Handlungsoption die maximal mögliche Differenz zu diesem Wert berechnet wird. Das maximal mögliche Bedauern, ist dann das Zeilenmaximum. \\
\textbf{Kritik:} Bezüglich der Bedauernsmatrix beruht die Entscheidung auf den Prinzipien des Minimax-Kriteriums. Daher werden auch bei diesem Kriterium nicht alle Informationen verwendet, und es drückt letztlich eine pessimistische Haltung aus. 
\begin{example} \\
	\begin{tabular}{|l|l|l|l|l||l|l|l|l|l|}
		\hline
		& $z_1$ & $z_2$ & $z_3$ & $z_4$ & $z_1$ & $z_2$ & $z_3$ & $z_4$ & Maximum\\ \hline
		$a_1$ & \textbf{60} & 30 & 50 & 60 & 60 - 60 = 0 & 100 - 30 = 70 & 120 - 50 = 70 & 140 - 60 = 80 & \textbf{80} \\ \hline
		$a_2$ & 10 & 10 & 10 & \textbf{140} & 60 - 10 = 50 & 100 - 10 = 90 & 120 - 10 = 110 & 140 - 140 = 0 & 110 \\ \hline
		$a_3$ & -30 & \textbf{100} & \textbf{120} & 130 & 60 - -30 = 90 & 100 - 100 = 0 & 120 - 120 = 0 & 140 - 130 = 10 & 90 \\ \hline		
	\end{tabular}
\end{example}

\subsubsection{Laplace-Kriterium}
\textbf{Regel:} Beim Laplace-Kriterium wird für jede Alternative der Durchschnitt aller Umweltzustände gebildet. \\
\textbf{Intention:} \aszeichen{Prinzip des mangelnden Grundes} d.h. da es keinen Grund gibt, dass ein Umweltzustand wahrscheinlicher ist als ein anderer wird davon ausgegangen, dass alle Zustände gleich wahrscheinlich sind.\\
\textbf{Kritik:} Es lässt sich festhalten, dass wahrscheinlich die meisten Alternative $a_3$ gewählt hätten und somit das Laplace-Kriterium verglichen mit den anderen Regeln realitätsnäher ist. Die Entscheidung wird beeinflusst durch das Hinzufügen praktisch identischer Spalten, d.h. für tatsächlich gleichwahrscheinliche Umweltzustände. Das Kriterium ist nur für risikoneutrale Entscheidungsträger rational, da Ausreisser gleichgewichtet werden.
\begin{example} \\
	\begin{tabular}{|l|l|l|l|l|l|}
		\hline
		& $z_1$ & $z_2$ & $z_3$ & $z_4$ & $\Phi(a_i)$ \\ \hline
		$a_1$ & 60 & 30 & 50 & 60 & (60 + 30 + 50 + 60) / 4 = 50 \\ \hline
		$a_2$ & 10 & 10 & 10 & 140 & (10 + 10 + 10 + 140) / 4 = 42.5 \\ \hline
		$a_3$ & -30 & 100 & 120 & 130 & (-30 + 100 + 120 + 130) / 4 = \textbf{80} \\ \hline		
	\end{tabular}
\end{example}

\subsection{Entscheidung unter Risiko}
\subsubsection{$\mu$-Kriterium/Bayes-Regel}
\textbf{Erwartungswert:} Der Erwartungswert ist der Wert, der sich als Mittelwert ergibt, wenn man die Situation unendlich oft wiederholen würde. Der Erwartungswert einer Alternative berechnet man, indem die Ergebniswerte mit den der dazugehörigen Wahrscheinlichkeit multipliziert und die sich so ergebenden Werte addiert werden. \\
\textbf{risikoscheu:} Der Entscheidungsträger bevorzugt einen niedrigeren Erwartungswert, wenn dieser mehr Sicherheit bringt.\\
\textbf{risikoneutral:} Dem Entscheidungsträger ist die Schwankung der Ergebnisse egal, er orientiert sich nur nach dem Erwartungswert.\\
\textbf{risikofreudig:} Der Entscheidungsträger verzichtet auf einen höheren Erwartungswert zugunsten einer grösseren Streuung der Ergebnisse.\\
\textbf{Fazit:} Für risikoneutrale Entscheidungsträger ist eine Entscheidung nach dem $\mu$-Kriterium rational. Darüber hinaus liefert das $\mu$-Kriterium bei oft wiederholten Entscheidungen bzw. Entscheidungen, deren Tragweite in Relation zum eigenen Vermögen nur klein ist, brauchbare Ergebnisse.
\begin{example} \\
	\begin{tabular}{|l|l|l|l|l|}
		\hline
		& $z_1$ & $z_2$ & $z_3$ & $\mu$ (Erwartungswert) \\
		& $p_1$ = 0.5 & $p_2$ = 0.2 & $p_3$ = 0.3 & \\ \hline
		$a_1$ & 30 & 20 & 20 & 0.5 * 30 + 0.2 * 20  + 0.3 * 20 = 25 \\ \hline
		$a_2$ & 140 & -40 & -30 & 0.5 * 140 + 0.2 * (-40) + 0.3 * (-30) = \textbf{53} \\ \hline
		$a_3$ & 40 & 10 & 60 & 0.5 * 40 + 0.2 * 10 + 0.3 * 60 = 40 \\ \hline		
	\end{tabular}
\end{example}

\subsubsection{Entscheidungsbaum}
Entscheidungsbäume sind geordnete, gerichtete Bäume, die der Darstellung von Entscheidungsregeln dienen. Die grafische Darstellung
als Baumdiagramm veranschaulicht hierarchisch aufeinanderfolgende Entscheidungen. \\
Ein Entscheidungsbaum besteht immer aus ...
\begin{compactitem}
	\item ... einem Wurzelknoten
	\item ... beliebig vielen inneren Knoten (logische Regel)
	\item ... mindestens zwei Blättern (Antwort auf Entscheidungsproblem)
\end{compactitem}
\textbf{Darstellung:}
\begin{compactenum}
	\item Die Zeit in einer Entscheidungsbaum-Darstellung läuft von	links nach rechts.
	\item Ereignisknoten werden als Kreise dargestellt,	Entscheidungsknoten als Quadrate.
	\item Die Kanten aus einem Entscheidungsknoten werden mit den	möglichen Entscheidungen angeschrieben. Alle möglichen Entscheidungen müssen vorkommen.
	\item  Die Kanten aus einem Ereignisknoten werden mit den möglichen	Ereignissen angeschrieben. Alle möglichen Ereignisse müssen
	vorkommen.
\end{compactenum}
\textbf{Entscheiden:}
\begin{compactenum}
	\item Entscheidungsbaum quantifizieren:
	\begin{compactitem}
		\item Bei jedem Ereignisknoten: Wahrscheinlichkeiten definieren
		\item An jedem freien Kantenende: CHF anschreiben
		\item Ein Ereignisknoten erhält als \aszeichen{expected monetary value} (EMV) das wahrscheinlichkeits-gewichtete Mittel der EMVs der unmittelbar rechts liegenden Knoten.
		\item Ein Entscheidungsknoten erbt den grössten EMV der unmittelbar rechts liegenden Knoten.
	\end{compactitem}
	\item Im Entscheidungsknoten wird jene Kante gewählt, die den höchsten Erwartungswert aufweist.
\end{compactenum}
\begin{example}
	Soll ein Angebot von CHF 10'000 angenommen werden, wenn bei seiner Ablehnung ein nächstes (letztes) Angebot auftauchen wird, das
	\begin{compactitem}
	\item mit 60\% Wahrscheinlichkeit CHF 15'000
	\item mit 30\% Wahrscheinlichkeit CHF 8'000
	\item und mit 10\% nichts wert ist?
	\end{compactitem}
	\begin{tikzpicture}[node distance = 0.6cm and 2.5cm]
		\node (resultA) [answer] {A: CHF 10'000};
		\node (resultB) [answer, below=of resultA] {B: CHF 15'000};
		\node (resultC) [answer, below=of resultB] {C: CHF 8'000};
		\node (resultD) [answer, below=of resultC] {D: CHF 0};
		\node (waitForB) [event, left=of resultC, label=north:{CHF 11'400}] {Alternative B};
		\node (J) [decision, left=of waitForB] {J};
		\draw (waitForB) -- node[anchor=south]{0.6} (resultB.west);
		\draw (waitForB) -- node[anchor=south]{0.3} (resultC);
		\draw (waitForB) -- node[anchor=south]{0.1} (resultD.west);
		\draw (J) -- node[anchor=south]{Warten auf B} (waitForB);
		\draw (J) |- node[anchor=north west]{A annehmen} (resultA);
	\end{tikzpicture}\\
	\aszeichen{Warten auf B} hat mit CHF 11'400 einen höheren Erwartungswert (EMV) als A.
\end{example}

\subsubsection{Bernoulli-Prinzip}
\textbf{Idee:} Die Grundidee besteht darin, den Erwartungswert nicht von den Ergebniswerten sondern von den Nutzenwerten zu berechnen und diesen dann als Präferenzwert zu betrachten. Zur Bestimmung der Nutzenfunktion kann die Bernoulli-Befragung
dienen, die eine Risiko-Nutzen-Funktion (RNF) oder kurz Nutzenfunktion $u(e_{ij})$ herausgibt.
\begin{example}
	Folgende Nutzenfunktion sei gegeben: $u(e_{ij}) = 300 * e_{ij} - e_{ij}^2$ \\
	\begin{tabular}{|l|l|l|l||l|l|l||l|}
		\hline
		& $z_1$ & $z_2$ & $z_3$ & $u(e_{i1})$ & $u(e_{i2})$ & $u(e_{i3})$ & $E(u(e_{ij}))$ \\
		& $p_1$ = 0.5 & $p_2$ = 0.2 & $p_3$ = 0.3 & $p_1$ = 0.5 & $p_2$ = 0.2 & $p_3$ = 0.3 & \\ \hline
		$a_1$ & 30 & 20 & 20 & 8'100 & 5'600 & 5'600 & 0.5*8'100+0.2*5'600+0.3*5'600=6'850\\ \hline
		$a_2$ & 140 & -40 & -30 & 22'400 & -13'600 & -9'600 & 0.5*22'400+0.2*(-13'600)+0.3*(-9'900)=5'510\\ \hline
		$a_3$ & 40 & 10 & 60 & 10'400 & 2'900 & 14'400 & 0.5*10'400+0.2*2'900+0.3*14'400=\textbf{10'100}\\ \hline
	\end{tabular} \\ \ \\
	Alternative 3 hat den höchsten Erwartungsnutzen. Da der Entscheider auf den hohen Erwartungswert der Alternative 2 ($\mu$  = 53) zugunsten der Alternative 3 ($\mu$ = 40) verzichtet, weil die Ergebnisse bei der Variante 2 unsicherer sind, ist er risikoscheu.
\end{example}
Wähle die Alternative, für die sich der grösste Erwartungsnutzen ergibt.
\section{Simulation}
\subsection{Monte Carlo Simulation}
Monte Carlo Simulation ist eine numerische Methode für statistische Simulation, die Sequenzen von Zufallszahlen benutzt, um die Simulation durchzuführen.

\textbf{Eigenschaften:}
\begin{compactitem}
	\item Monte Carlo Simulation ist ein kräftiges Instrument, um komplexe statistische Analysen durchzuführen und Wahrscheinlichkeiten und Verteilungen zu schätzen.
	\item Es verlangt ein Systemmodell (quantitative Systembeschreibung).
	\item Es werden (virtuelle) Experimente mit dem System ausgeführt, um Schlussfolgerungen bzgl. deren Verhalten zu ziehen.
\end{compactitem}

\begin{example}
	Berechne den Wert von $\pi$
	\begin{multicols}{5}
		Fläche Rechteck: $(2r)^2$ \\
		Fläche Kreis: $\pi r^2$ \\
		$\frac{\text{Fläche Rechteck}}{\text{Fläche Kreis}}$: $\frac{4}{\pi}$ \\
		$\pi$: $4 * \frac{\text{Fläche Kreis}}{\text{Fläche Rechteck}}$ \\
		$\pi$: $4 * \frac{\text{Punkte im Kreis}}{\text{Punkte im Rechteck}}$	
	\end{multicols}
	\begin{minipage}[h]{0.825\textwidth}
		\begin{lstlisting}[mathescape=true, tabsize=2]
N Punkte $X_i$ = -1 + 2$A_i$ und
N Punkte $Y_i$ = -1 + 2$B_i$ mit A, B Sequenzen von unabhaengigen Zufallszahlen
K = 0
for i = 1 : N
	if ($X_i^2$ + $Y_i^2$ < 1)
		K = K + 1
	end
end
$\pi$ = 4 * K / N
		\end{lstlisting}
	\end{minipage}
	\begin{minipage}[h]{0.175\textwidth}
		\includegraphics[width=1\textwidth]{pictures/montecarlopi}
	\end{minipage}
\end{example}

\subsection{Diskrete Ereignissimulation}
\subsubsection{Warteschlangentheorie}
Die Warteschlangentheorie beschäftigt sich mit der mathematischen Analyse von Systemen, in denen Aufträge von Bedienungsstationen bearbeitet werden und gibt Antwort auf die Fragen nach den charakteristischen Grössen, wie der Stabilität des Wartesystems, der Anzahl Kunden im System, ihrer Wartezeit etc. Sie unterstützt unter anderem Führungsentscheidungen über den Personaleinsatz und den Abfertigungsprozess. Ihre Anwendung reicht von Telekommunikationssystemen, Verkehrssystemen über Logistik bis zu Fertigungssystemen.
\begin{example}\\
	\begin{tabular}{|l|l|l|}
		\hline
		\textbf{System} & \textbf{Bedienungsstation} & \textbf{Aufträge} \\ \hline
		Bank & Schalter & Kundenbesuche \\ \hline
		Spital & Ärzte, Pflegende, Betten & Patientinnen und Patienten \\ \hline
		Rechner & CPU, I/O-Geräte & Jobs \\ \hline
		Fertigung & Maschinen, Operateure & Bauteile \\ \hline
		Rettungsdienst & Rettungsfahrzeuge, Notfallärzte & Patientinnen und Patienten \\ \hline		
	\end{tabular}
\end{example}

\subsubsection{Hauptmerkmale}
\begin{compactitem}
	\item Anwesenheit stochastischer Prozesse
	\item Zeit(-ablauf) spielt eine wichtige Rolle.
	\item Wertveränderungen der Variablen werden verursacht durch Ereignisse und treten nur an diskreten Zeitpunkten auf.
\end{compactitem}

\begin{multicols}{2}
	\textbf{Vorteile:}
	\begin{compactitem}
		\item Kostengünstiges und sicheres Experimentierfeld
		\item Ermöglicht Analyse komplexer Systeme durch hohen Detaillierungsgrad
		\item Ermöglicht Animation und steigert somit Systemverständnis
	\end{compactitem}
	\textbf{Nachteile:}
	\begin{compactitem}
		\item Erfordert einen hohen initialen Zeitaufwand
		\item Bau eines Simulationsmodells ist relativ fehleranfällig
		\item Interpretation der Analysedaten ist anspruchsvoll
	\end{compactitem} \ \\
\end{multicols}

\subsubsection{M/M/1 Modell}
\begin{multicols}{3}
	\begin{compactitem}[$\bullet$]
		\item M-Elemente betreten Warteschlange
		\item M-Elemente sind in Warteschlange
		\item 1-Element verlässt Warteschlange
	\end{compactitem}
\end{multicols}
\begin{example}\\
	\includegraphics[width=0.7\textwidth]{pictures/mm1modell}
\end{example}
\todo{Folie 32 - Diagramme allenfalls noch einfügen}

\subsubsection{Kernelemente}
\textbf{Items:} Objekte, die durch das System fliessen (z.B. Patienten, Erstzteile, ...). Items haben meistens Attribute (Ankunftszeit, Bearbeitungsstartzeit, Prioritätsklasse, usw.). \\
\textbf{Systemzustand:} Der Zustand $Z(t)$ gibt eine vollständige Beschreibung des Systems am Zeitpunkt $t$. Der Systemzustand ist meistens ein (mehrdimensionaler) Vektor und dient u.a. der Berechnung der Zielgrössen. \\
\textbf{Simulationsuhr:} Die Simulationsuhr hält die virtuelle Zeit im Simulationsmodell fest. \\
\textbf{Ereignis:} Jedes Ereignis hat eine Wirkung, die zum Auftretenszeitpunkt ausgeübt wird. Die Wirkung besteht aus einer Änderung des Systemzustands und/oder einer Änderung der Future-Event-List. In der Zeit zwischen zwei nachfolgenden Ereignissen passiert im System nichts (der Systemzustand ändert sich nicht). Aus diesem Grund kann die Simulationsuhr von Ereignis zu Ereignis springen. \\
\textbf{Future Event List:} Dynamische Liste, die 0 oder mehr (Zeitpunkt-, Ereignis-) Paare enthält; Sie umfasst alle Ereignisse, die zum aktuellen Zeitpunkt bekannt sind. Während eines Simulationslaufs werden ständig neue Ereignisse hinzugefügt und verarbeitete Ereignisse gelöscht.

\subsubsection{Spezifikation eines DE Modells}
DE Modelle werden spezifiziert durch
\begin{compactenum}
	\item Flussdiagramme
	\item Ereignisdiagramme
\end{compactenum}

\begin{example}[Beispiel für nachfolgende Kapitel 4.2.6 - 4.2.7]
	Wir betrachten eine Maschine, worauf genau ein Produkt hergestellt wird. Der Ankunftsprozess der Produktionsaufträge ist poissonverteilt mit dem Parameterwert $\lambda$. Bearbeitungszeiten sind exponentialverteilt mit dem Parameterwert $\mu$.\\
	Bei der Produktion können Fehler auftreten. Beim Maschinenausgang werden die Produkte visuell kontrolliert. Die Wahrscheinlichkeit eines Fehlers ist $\epsilon$. Misslungene Produktionsaufträge müssen wiederholt werden. \\
	\textbf{Items:} Produktionsaufträge mit Attributen: Ankunftszeit, Bearbeitungsstartzeit, Bearbeitungszeit, Anzahl Fehlversuche, etc. \\
	\textbf{Zustand:} $Z$ = (Anzahl) Produktionsaufträge im System \\
	\textbf{Ereignisse:} $e_1$ = Ankunft eines Produktionsauftrags; $e_2$ = Bearbeitung auf der Maschine abgeschlossen
\end{example}

\subsubsection{Flussdiagramm}
\textbf{Grundbausteine:}
\begin{multicols}{5}
	Quelle und Senke: \\ \\
	Fluss: \\ \\
	Weiche: \\ \\
	Warteschlange: \\ \\
	Aktivität oder Ressource:
\end{multicols}	
\begin{multicols}{5}
	\includegraphics[width=0.15\textwidth]{pictures/fluss_quelle_senke}\\ \\
	\includegraphics[width=0.15\textwidth]{pictures/fluss_fluss}\\
	\includegraphics[width=0.15\textwidth]{pictures/fluss_weiche}\\ 
	\includegraphics[width=0.1\textwidth]{pictures/fluss_warteschlange}\\ 
	\includegraphics[width=0.15\textwidth]{pictures/fluss_aktivitaet}
\end{multicols}
\begin{multicols}{2}
	\begin{multicols}{2}
		Gabelung (konvergent):\\
		Gabelung (divergent):
	\end{multicols}
	Lager:
\end{multicols}
\begin{multicols}{2}
	\begin{multicols}{2}
		\includegraphics[width=0.2\textwidth]{pictures/fluss_gabelung1}\\ 
		\includegraphics[width=0.2\textwidth]{pictures/fluss_gabelung2}
	\end{multicols}
	\includegraphics[width=0.4\textwidth]{pictures/fluss_lager}
\end{multicols}
\begin{example} \\
	\includegraphics[width=0.5\textwidth]{pictures/flussdiagramm}
\end{example}

\subsubsection{Ereignisdiagramm}
\begin{example} 
	\begin{multicols}{2}
		\textbf{Ereignisdiagramm $e_1$:} \\
		\includegraphics[width=0.5\textwidth]{pictures/ereignisdiagramm1}
		\textbf{Ereignisdiagramm $e_2$:} \\
		\includegraphics[width=0.5\textwidth]{pictures/ereignisdiagramm2}
	\end{multicols}
\end{example}

\subsubsection{Simulationskern / Ereignismanager}
\includegraphics[width=0.5\textwidth]{pictures/ereignismanager}

\subsection{Wahrscheinlichkeitsverteilungen}	
\textbf{Vorteile:} 	
\begin{compactitem}
	\item Kompakte Formulierung (meistens 1 oder 2 Parameter)
	\item Leicht anzupassen (über die geringe Anzahl Parameter)
	\item Grosser Bereich an ausgewürfelten Zufallszahlen
\end{compactitem}	
\textbf{Nachteile:}
\begin{compactitem}
	\item Beschränkte Anzahl Parameter, um die Verteilung in die gewünschte Richtung anzupassen. Manchmal einfach nicht möglich, die komplexe Wirklichkeit in einer Verteilung mit 1 oder 2 Parametern abzubilden.
	\item Verteilung repräsentiert die Wirklichkeit oft nicht gut in Extremsituationen.
\end{compactitem}

\begin{multicols}{3}
	\subsubsection{Uniformverteilung} 
	$X \sim U[0, 200]$ \\
	\includegraphics[width=0.25\textwidth]{pictures/uniformverteilung} 
	\subsubsection{Normalverteilung} 
	$X \sim N(100, 10'000)$ \\
	\includegraphics[width=0.25\textwidth]{pictures/normalverteilung} 
	\subsubsection{Empirische Verteilung}
	\begin{tabular}{|l|l|}
		\hline
		\textbf{Zufallszahl} & \textbf{Wahrscheinlichkeit} \\ \hline
		1 & 0.4 \\ \hline
		2 & 0.3 \\ \hline
		4 & 0.1 \\ \hline
		8 & 0.15 \\ \hline
		16 & 0.05 \\ \hline
	\end{tabular}
\end{multicols}

\subsubsection{Empirische Verteilung - Vor- / Nachteile}
\begin{compactitem}
	\item Basiert auf realen Daten
	\item Alle möglichen Formen sind möglich
\end{compactitem}	
\textbf{Nachteile:}
\begin{compactitem}
	\item Es können keine Zufallszahlen erzeugt werden, die nicht schon in der Vergangenheit aufgetreten sind. Daher braucht man eine lange	Historie.
	\item Empirische Verteilungen kann man nicht kompakt auf Papier darstellen.
\end{compactitem}

\subsubsection{Parameterschätzungen Gefahren}
\begin{multicols}{2}
	\begin{compactitem}
		\item Zu wenig Datenmaterial
		\item Nicht-repräsentative Vergangenheitsdaten
		\item Unberücksichtigte Ausreisser
		\item Unerkannte Zeitmuster (Sommer / Winter, Mo-Fr versus Sa-So)
		\item Falsche Interpretation des vorhandenen Datenmaterials
	\end{compactitem}
\end{multicols}
	
	
\section{Systemdenken \& Systemdynamik}
\subsection{Einführung}
\begin{multicols}{2}
	\subsubsection{Kausales Modell}
	\includegraphics[width=0.5\textwidth]{pictures/kausales_modell}

	\subsubsection{Polarität einer Kausalbeziehung}
	\textbf{Positive Polarität:} Steigt (fällt) Ursache, dann steigt (fällt) Wirkung \\
	\includegraphics[width=0.35\textwidth]{pictures/positive_polaritaet}\\
	\textbf{Negative Polarität:} Steigt (fällt) Ursache, dann fällt (steigt) Wirkung \\
	\includegraphics[width=0.35\textwidth]{pictures/negative_polaritaet}
\end{multicols}

\begin{multicols}{3}
	\subsubsection{Rückkopplung}
	Gerichteter Kreis von Kausalbeziehungen. \\
	\includegraphics[width=0.25\textwidth]{pictures/rueckkopplung} \\
	
	\subsubsection{Selbstverstärkender Loop}
	Die Rückkoppelung verstärkt einen anfänglichen Effekt. \\
	\includegraphics[width=0.25\textwidth]{pictures/selbstverstaerkender_loop}
	
	\subsubsection{Ausgleichender Loop}
	Die Rückkoppelung wirkt	einem anfänglichen Effekt entgegen.\\
	\includegraphics[width=0.25\textwidth]{pictures/ausgleichender_loop}
\end{multicols}	

\begin{multicols}{2}
	\subsubsection{Stock-and-Flow Diagramm}
	\includegraphics[width=0.5\textwidth]{pictures/stock_and_flow_diagramm}
	
	\subsubsection{Systemdynamisches Simulationsmodell}
	\includegraphics[width=0.5\textwidth]{pictures/systemdynamisches_simulationsmodell}
\end{multicols}	

\subsection{Grundideen}
\subsubsection{Ziele}
\begin{compactitem}
	\item \textbf{Verständnis, nicht Prognose} \\
	The goal of a system dynamics policy study is understanding - understanding the interactions in a complex system that are 	conspiring to create a problem, and understanding the structure	and dynamic implications of policy changes intended to improve	the system's behaviour (Richardson 1991: 164)
	\item \textbf{Modellieren, um zu lernen} \\
	Therefore, the primary goal is not to build the model of the system, but rather to get a group engaged in building a system dynamics model of a problem in order to see to what	extent this process might be helpful to increase problem understanding and to devise courses of action to which team members feel committed (Vennix 1996: 3).
\end{compactitem}

\subsubsection{Darstellungsformen}
\begin{compactenum}
	\item \textbf{Causal Loop Diagramme (\aszeichen{Kausaldiagramme})}
	\begin{compactitem}
		\item Kommunikations-Tool für ein gemeinsames mentales Modell und generelle Diskussionen
	\end{compactitem}
	\item \textbf{Stock-and-Flow Diagramme (\aszeichen{Flussdiagramme})}
	\begin{compactitem}
		\item Präzise Darstellung von Lagern (Stocks) und Raten (Flows)
	\end{compactitem}
	\item \textbf{Stock-and-Flow Diagramme mit Formeln}
	\begin{compactitem}
		\item Formulierung sämtlicher Flüsse als Formeln
		\item Erlaubt quantitative Simulation
	\end{compactitem}
\end{compactenum}

\subsubsection{Computersimulation}
\begin{compactitem}
	\item Die Dynamik von Feedbackmodellen korrekt interpretieren
	\item Nicht antizipierte Nebenwirkungen entdecken
	\item Modellanalyse
	\begin{compactitem}
		\item Sensitivitätsanalyse von Parametern und funktionalen Beziehungen
		\item Einfluss unterschiedlicher Modellformulierungen
		\item Massnahmenanalyse (Policy Analysis)
	\end{compactitem}
\end{compactitem}

\subsection{Akkumulation}
Grundlage zum Verstehen von Spät- und Rückwirkungen. Akkumulation zu verstehen ist eine Voraussetzung, um Systemverhalten zu verstehen.

\subsubsection{Definitionen}
Der Verlauf einer Bestandesgrösse über die Zeit ist durch die untenstehende hydraulische Metapher definiert - bzw. durch die equivalenten mathematischen Definitionen.\\
\begin{multicols}{2}
	\textbf{Bestandes- und Flussdiagramm (Stock-and-Flow-Diagramm):} \\
	\includegraphics[width=0.5\textwidth]{pictures/badewanne_stock} \\
	\textbf{Integralgleichung:} \\
	$Stock(t) = \integral{t_0}{t}{[Inflow(s)-Outflow(s)]}{s}+Stock(t_0)$\\
	\textbf{Differentialgleichung:} \\
	$\differential{Stock(t)}{t} = Inflow(t) - Outflow(t)$ \\
	\textbf{Metapher aus der Hydraulik:} \\
	\includegraphics[width=0.35\textwidth]{pictures/badewanne}
\end{multicols}

\begin{multicols}{2}
	\subsubsection{Bestandesgrössen (Stocks)}
	\begin{compactitem}
		\item sind Zustandsgrössen, die “Erinnerung” eines Systems
		\item müssen mit einem Anfangswert charakterisiert werden
		\item ändern sich ausschliesslich durch Zu- und Abfluss
		\item Wert verändert sich nicht \textless-\textgreater\ Zufluss = Abfluss
		\item Wert wächst \textless-\textgreater\ Zufluss \textgreater\ Abfluss
		\item Wert fällt \textless-\textgreater\ Zufluss \textless\ Abfluss
		\item haben Masseinheiten wie Liter, CHF, Meter
	\end{compactitem}
	
	\subsubsection{Flussgrössen (Flows)}
	\begin{compactitem}
		\item sind Änderungsraten von Bestandesgrössen
		\item haben Masseinheiten wie Liter/Stunde, CHF/Jahr, Meter/Sekunde
	\end{compactitem} \ \\ \ \\ \ \\
\end{multicols}

\subsection{Causal Loop Diagramme}
\subsubsection{Vergleich CLD und Stock-and-Flow Diagramm}
\begin{multicols}{2}
	\textbf{Stock-and-Flow Diagramm:} \\
	\includegraphics[width=0.5\textwidth]{pictures/vergleich_stock} \\ \\
	\textbf{Causal Loop Diagramm:} \\
	\includegraphics[width=0.5\textwidth]{pictures/vergleich_cld} 
\end{multicols}	

\subsubsection{Unmittelbare und akkumulierende Kausalbeziehungen}
Werden im CLD graphisch nicht unterschieden.
\begin{multicols}{2}
	\textbf{Unmittelbar:} \\
	Steigt (fällt) A, dann steigt (fällt) B \\
	\includegraphics[width=0.35\textwidth]{pictures/unmittelbar_1} \\
	Steigt (fällt) A, dann fällt (steigt) B \\
	\includegraphics[width=0.35\textwidth]{pictures/unmittelbar_2} \\
	\textbf{Akkumulierend:} \\
	A wird zu B addiert, $B=\integral{t_0}{t}{A}{d}+A_0$\\
	\includegraphics[width=0.35\textwidth]{pictures/akkumulierend_1} \\
	A wird von B subtrahiert, $B=\integral{t_0}{t}{-A}{d}+A_0$\\
	\includegraphics[width=0.35\textwidth]{pictures/akkumulierend_2}
\end{multicols}	

\subsubsection{Möglichkeiten}
\begin{compactitem}
	\item sich in einem komplexen System Übersicht verschaffen, System verstehen
	\item Steuermöglichkeiten (\aszeichen{Stellhebel}) erkennen
	\item Suche nach Neben- und Rückwirkungen von scheinbar zielführenden Massnahmen erleichtern
	\item Zielgrössen identifizieren, Zielkonflikte offen legen
	\item Diskussionsqualität verbessern: präzise erklären, wovon die Rede ist
\end{compactitem}

\newpage

\subsubsection{Regeln für gute CLDs}
\begin{multicols}{2}
	\begin{compactenum}
		\item (Variablennamen) Substantive als Variablennamen wählen
		\item (Variablennamen) Keine Richtung der Dynamik durch Variablennamen vorgeben \\
		\includegraphics[width=0.45\textwidth]{pictures/regel_2}
		\item (Variablennamen) Der Wert von Variablen muss grösser oder kleiner werden können - der Name bezeichnet im Idealfall ein Kontinuum \\
		\includegraphics[width=0.45\textwidth]{pictures/regel_3}
		\item (Kausalbeziehungen) Kausalität, nicht Korrelation modellieren
		\begin{compactitem}
			\item Nur Kausalitäten aufzeichnen, die Sie überzeugen
			\item Relevante kausale Erklärungen zu finden, ist ein kreativer Akt und erfordert den Austausch zwischen den beteiligten Personen (Statistik	reicht nicht)
		\end{compactitem}
		\includegraphics[width=0.45\textwidth]{pictures/regel_4}
		\item (Kausalbeziehungen) Kausalbeziehungen mit eindeutiger Polarität modellieren: Ist keine Polarität zuweisbar, dann muss die	Kausalbeziehung, bzw. unterliegende Struktur besser	analysiert werden. \\
		\includegraphics[width=0.225\textwidth]{pictures/regel_5} 
		\includegraphics[width=0.225\textwidth]{pictures/regel_5_2}
		\item (Kausalbeziehungen) Kein \aszeichen{Einkaufszettel} (Daumenregel: max. 3 Ursachen für	eine Wirkung) \\
		\includegraphics[width=0.45\textwidth]{pictures/regel_6} \\ \\ \\ \\ \\
		\item (Kausalbeziehungen) Vermeiden Sie Redundanz \\
		\includegraphics[width=0.225\textwidth]{pictures/regel_7} 
		\includegraphics[width=0.225\textwidth]{pictures/regel_7_2}
		\item (Kausalbeziehungen) Stellen Sie bei jeder Entscheidungsregel die Frage: Gibt es auch unerwünschte Wirkungen? \\
		\includegraphics[width=0.225\textwidth]{pictures/regel_8} 
		\includegraphics[width=0.225\textwidth]{pictures/regel_8_2}
		\item (Kausalbeziehungen) Markieren Sie wichtige Zeitverzögerungen mit // \\
		\includegraphics[width=0.225\textwidth]{pictures/regel_9}
		\item (Loops) ... und schliessen Sie die Loops \\
		\includegraphics[width=0.225\textwidth]{pictures/regel_10}
		\item (Loops) Validieren Sie Rückkoppelungen: Welche Geschichte erzählt ein	Loop? Plausibilisieren Sie die mathematische Polarität eines Loops mit Ihrer Geschichte. \\
		\includegraphics[width=0.225\textwidth]{pictures/regel_11}
		\item (Gesamtmodell) Fokussieren Sie auf den Modellzweck:
		\begin{compactitem}
			\item Welche Zielgrössen müssen zwingend modelliert werden?
			\item Welche Geschichten sind Ihnen - und anderen beteiligten Personen - besonders wichtig?
		\end{compactitem}
	\end{compactenum}
\end{multicols}	

\subsubsection{Verhaltensmuster}
Dynamische Systeme zeigen oftmals eines der folgenden charakteristischen Verhaltensmuster auf:
\begin{multicols}{3}
	\textbf{Exponentielles Wachstum (exponential growth):} \\
	\includegraphics[width=0.3\textwidth]{pictures/verhalten_1} \\
	\textbf{Nach einem Ziel strebend (goal seeking):} \\
	\includegraphics[width=0.3\textwidth]{pictures/verhalten_2} \\ \\
	\textbf{Oszillation (oscillation):} \\
	\includegraphics[width=0.3\textwidth]{pictures/verhalten_3}
\end{multicols}	
\begin{multicols}{3}
	\textbf{S-Kurve (S-shaped growth):} \\
	\includegraphics[width=0.3\textwidth]{pictures/verhalten_4} \\ \\
	\textbf{Wachstum mit Überschiessen (growth with overshoot):} \\
	\includegraphics[width=0.3\textwidth]{pictures/verhalten_5} \\
	\textbf{Überschiessen und Kollaps (overshoot and collapse):} \\
	\includegraphics[width=0.3\textwidth]{pictures/verhalten_6} 
\end{multicols}	

\subsubsection{Strukturen Verhaltensmuster}
\begin{multicols}{2}
	\textbf{Exponentielles Wachstum:} \\
	\includegraphics[width=0.3\textwidth]{pictures/struktur_1} \\ 
	\textbf{Nach einem Ziel strebend:} \\
	\includegraphics[width=0.45\textwidth]{pictures/struktur_2} 
\end{multicols}	
\begin{multicols}{2}
	\textbf{Oszillation:} \\
	\includegraphics[width=0.3\textwidth]{pictures/struktur_3} \\
	\textbf{S-Kurve:} \\
	\includegraphics[width=0.3\textwidth]{pictures/struktur_4} 
\end{multicols}	
\begin{multicols}{2}
	\textbf{Wachstum mit Überschiessen:} \\
	\includegraphics[width=0.3\textwidth]{pictures/struktur_5} \\ \\
	\textbf{Überschiessen und Kollaps:} \\
	\includegraphics[width=0.45\textwidth]{pictures/struktur_6} 
\end{multicols}	

\subsubsection{System-Archetypen}
Muster und Strukturen, die in verschiedenen Kontexten in ähnlicher Weise auftreten.\\
\textbf{Vorteil}: Muster erkennen hilft relevante Dynamik-Probleme zu identifizieren.\\
\textbf{Achtung}: Real-World-Probleme benötigen trotzdem noch eine sorgfältige, spezifische Untersuchung.\\
\begin{multicols}{3}
	\textbf{Prozess balancieren mit Verzögerung (balancing process with delay):}\\
	\includegraphics[width=0.8\linewidth]{pictures/archetype1}\\
	\vfill\null
	\columnbreak
	\textbf{Limiten beim Wachstum (limits to growth):}\\
	\includegraphics[width=\linewidth]{pictures/archetype2}\\
	\vfill\null
	\columnbreak
	\textbf{Fehlgeschlagene Lösungen (fixes that fail):}\\
	\includegraphics[width=0.8\linewidth]{pictures/archetype3}\\\
\end{multicols}
\begin{multicols}{3}
	\textbf{Verlagerung der Last (shifting the burden):}\\
	\includegraphics[width=\linewidth]{pictures/archetype4}\\
	\vfill\null
	\columnbreak
	\textbf{Erodierende Ziele (eroding goals)}\\
	\includegraphics[width=0.8\linewidth]{pictures/archetype5}\\
	\vfill\null
	\columnbreak
	\textbf{Eskalation (escalation):}\\
	\includegraphics[width=\linewidth]{pictures/archetype6}\\
\end{multicols}
\begin{multicols}{3}
	\textbf{Tragödie der Gemeingüter (tragedy of the commons):}\\
	\includegraphics[width=\linewidth]{pictures/archetype7}\\
	\vfill\null
	\columnbreak
	\textbf{Wachstum und Unterfinanzierung (growth and underinvestment):}\\
	\includegraphics[width=\linewidth]{pictures/archetype8}\\
	\vfill\null
	\columnbreak
	\textbf{Erfolg den Erfolgreichen (success to the successful)}\\
	\includegraphics[width=0.8\linewidth]{pictures/archetype9}\\
\end{multicols}

\subsubsection{Pfadabhängigkeit}
Ein Lock-in-Zustand liegt vor, wenn
\begin{compactitem}
	\item auf Grund von Entscheidungen in der Vergangenheit
	\item eine Änderung der aktuellen Situation mit hohen Kosten verbunden ist und
	\item deshalb oft unterbleibt.
\end{compactitem}
\begin{example}
	\begin{compactitem}
		\item billige Drucker und teure Farbpatronen
		\item Bonussysteme, Coop bis Lufthansa
		\item Beim Kauf eines Instruments wird die vorgängig bezahlte Miete teilweise angerechnet
	\end{compactitem}
\end{example}
\textbf{Linearer Polya-Prozess:} \\
\begin{tikzpicture}[node distance = 0.6cm and 0.5cm]	
	\node (n1) [decisionW] {Holzkugel ziehen};
	\node (n2) [decisionW, right=of n1] {Holzkugel zurücklegen und Holzkugel der gleichen Farbe	dazulegen};
	\draw [arrow] (n1) -- (n2);
	\draw [arrow] (n2.north) to[bend right] (n1.north);
\end{tikzpicture} \\
In einer Urne befinden sich eine blaue und eine grüne Kugel. Es wird blind eine Kugel herausgezogen. Die Wahrscheinlichkeit zur Ziehung einer Farbe beträgt 0.5. Die gezogene Kugel wird wieder zurückgelegt. Es wird nun eine weitere Kugel, welche die Farbe der soeben gezogenen aufweist, hinzugefügt. Es befindet sich nun eine Kugel mehr in der Urne als vor dem Zug. Wurde demnach eine blaue Kugel gezogen, beträgt die Wahrscheinlichkeit für die erneute Ziehung einer blauen Kugel 0.66. Dieser Vorgang ist pfadabhängig, da die Wahrscheinlichkeit, eine Kugel zu ziehen, welche eine gewünschte Farbe aufweist, mit der Anzahl an Kugeln jener Farbe zusammenhängt. Bereits der erste Zug hat eine hohe Relevanz, da die Anzahl an Kugeln evtl. noch gering ist und das Ziehen und das darauffolgende Hinzufügen einer Kugel auf den späteren Verlauf signifikante Auswirkungen hat. \\
\textbf{Lock-in Situation:}
\begin{compactitem}
	\item Ein und dasselbe System ist in unterschiedlichen Phasen sehr unterschiedlich empfindlich auf Störungen/Beeinflussungen.
	\item Bei bestimmten Systemzuständen reicht eine minimale Einflussnahme, um das	System in die gewünschte Richtung zu	beeinflussen. Solche Zustände liegen oft zu Beginn einer Entwicklung.
	\item In anderen Systemzuständen (meist später,	nach längerer Entwicklung), kann das System	nur noch mit grösstem Aufwand aus diesem Zustand gekippt werden (hohe Investition).
\end{compactitem}
\end{document}