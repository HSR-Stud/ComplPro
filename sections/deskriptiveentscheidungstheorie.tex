\section{Deskriptive Entscheidungstheorie}
\subsection{Aspekte des Entscheidens}
\begin{multicols}{2}
	\textbf{Rationales Entscheiden:} 
	\begin{compactitem}
		\item Entscheidungsprozess ist zielgerichtet und orientiert sich konsequent an Zielen
		\item Überlegungen basieren auf möglichst objektiven Informationen
		\item Entscheidungsprozess folgt systematischen Vorgehen und verwendet klare methodische Regeln
	\end{compactitem}

	\textbf{Intuitives Entscheiden:} 
	\begin{compactitem}
		\item Lassen sich nicht zwingend nachvollziehen
		\item Entscheidunglogik oftmals auch der entscheidenden Person unbekannt
		\item Keine quantitative Bewertung
	\end{compactitem} \ \\
\end{multicols}

\subsection{Mentales Modell}
Vereinfachte mentale Abbildung der subjektiv wahrgenommenen Aussenwelt mit dem Zweck,
\begin{compactitem}
	\item die durch die Sinnesorgane aufgenommene grosse Informationsmenge zu bewältigen.
	\item Entscheidungen schnell fällen zu können.
\end{compactitem}
Beispiele: Duschwasser einstelle, Strasse überqueren, Schuhe binden, ... \\
Diese mentalen Prozessmodelle sind unbewusst, mittels Erfahrung gewonnen und werden sofort angepasst. \\
Nutzen: Informationsverarbeitung, Effizienzsteigerung, Notwendig um komplexe Sachverhalte zu interpretieren

\subsubsection{Probleme}
\begin{compactenum}
	\item Aberglaube (Ticket-Automat im Bus)
	\item Vorurteile (Rosenhan-Experiment - Psychiatrie)
	\item Mentale Modelle können eine Eigendynamik entwickeln (selektive Suche nach Bestätigung)
	\item Konservativismus: Mentale Modelle entwickeln sich mit der	Erfahrung
	\item Scheinsicherheit: Wenn mentale Modelle nicht als solche wahrgenommen werden, werden sie nicht hinterfragt
\end{compactenum}
Mentale Modelle lassen sich nicht verifizieren!

\subsection{Kausalität vs. Korrelation}
\begin{multicols}{2}
	\textbf{Korrelation:}
	Zwei Grössen sind korreliert, wenn sich statistisch eine mathematische Abhängigkeit nachweisen lässt.
	\ \\ \\
	\textbf{Kausalität:}
	Zwischen zwei Grössen besteht ein kausaler Zusammenhang, wenn zwischen Ihnen eine Ursache-Wirkungsbeziehung besteht.
\end{multicols}

\subsection{Falsifikationismus}
Bei Eintreffen eines bestimmten experimentellen Befunds wird die Hypothese verworfen.\\
\textbf{Beispiel:} Alle Schwäne sind weiss (Hypothese). Noch so viele weisse Schwäne beweisen die Aussage nicht. Wissenschaftliche Arbeit: Suche nach einem allfälligen schwarzen Schwan.