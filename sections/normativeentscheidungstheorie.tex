\section{Normative Entscheidungstheorie}
\begin{example}[Beispiel für Kapitel 3.1 - 3.4]
	Angenommen ein Mineralölkonzern strebe nach Gewinnmaximierung und überprüfe die Eröffnung einer neuen Tankstelle. In Frage kommen zwei Standorte:
	\begin{multicols}{2}
		\begin{compactenum}
			\item Standort im Stadtzentrum (a1): 
			\begin{compactitem}
				\item Erwarteter Gewinn 125'000.- CHF 
				\item Erwarteter Umsatz 2'000'000.- CHF
			\end{compactitem}
			\item Standort am Stadtrand (a2): 
			\begin{compactitem}
				\item Erwarteter Gewinn 150'000.- CHF 
				\item Erwarteter Umsatz 1'800'000.- CHF
			\end{compactitem}
		\end{compactenum}
	\end{multicols}
\end{example}

\subsection{Entscheidung unter Unsicherheit}
\textbf{Hauptsächliche Beschäftigungsgebiet:} Unsicherheit und nicht der Zielkonflikt \\
\textbf{Umweltzustände:} Faktoren auf welche man selber aber keinen Einfluss hat, die aber die Entscheidung beeinflussen\\
Bei Entscheidungssituationen in denen eine bestimmte Entscheidung nicht mit einem einzigen möglichen Ereignis verbunden ist, wird von einer Entscheidung unter Unsicherheit gesprochen.

\subsection{Ergebnismatrix}
Darstellungsform für Entscheidungen unter Unsicherheit wobei die Handlungsalternativen ($a_i$) in den Zeilen und die verschiedenen Umweltzustände ($z_i$) in den Spalten dargestellt werden. 
\begin{example}
	Dem Mineralölkonzern ist bekannt, dass eine weiträumige Umgehungsstrasse geplant ist, die aber höchst umstritten ist.
	\begin{compactitem}
		\item Diese würden den Verkehr im Stadtzentrum nicht tangieren und auf CHF 125'000.- belassen.
		\item Am Stadtrand würde sich aber eine drastische Veränderung ergeben,	die den erwarteten Gewinn auf CHF 80'000.- reduzieren liesse.
	\end{compactitem}
	Welcher Standort wäre unter der Prämisse der Gewinnmaximierung nun zu präferieren? \\
	\begin{tabular}{|l|l|l|}
		\hline
		& $z_1$ (keine Umgehungsstrasse) & $z_2$ (Umgehungsstrasse) \\ \hline
		$a_1$ (Stadtzentrum) & 125'000.- = $e_{11}$ & 125'000.- = $e_{11}$ \\ \hline
		$a_2$ (Stadtrand) & 150'000.- = $e_{21}$ & 80'000.- = $e_{22}$ \\ \hline
	\end{tabular}
\end{example}

\subsubsection{Zustandsraum}
\begin{compactitem}
	\item Raum aller möglicher Entscheidungsergebnisse.
	\item Die Entscheidungsmatrix beschreibt die Entscheidungssituation nur dann hinreichend, wenn der Zustandsraum voll-ständig erfasst wird.
	\item In der Realität oft nicht gegeben.
\end{compactitem}

\subsection{Klassifikation}
\textbf{Entscheidung unter Risiko:} Falls den Umweltzuständen Eintrittswahrscheinlichkeiten zugeordnet werden können. \\
\textbf{Entscheidung unter Ungewissheit:} Falls den Umweltzuständen keine Wahrscheinlichkeiten zugeordnet werden können. \\
Wahrscheinlichkeiten müssen nicht mathematisch fundiert sein (objektiv), sondern können sich auch aus Sicht des Entscheiders ergeben (subjektiv).
\begin{example}
	Angenommen der Ölkonzern schätzt die Wahrscheinlichkeit für den Bau	der Umgehungsstrasse auf 30\%. \\
	\begin{tabular}{|l|l|l|}
		\hline
		Alternative & $z_1$ (keine Umgehungsstrasse) & $z_2$ (Umgehungsstrasse) \\ 
		& $p_1$ = 0.7 & $p_2$ = 0.3 \\ \hline
		$a_1$ (Stadtzentrum) & 125'000.- = $e_{11}$ & 125'000.- = $e_{11}$ \\ \hline
		$a_2$ (Stadtrand) & 150'000.- = $e_{21}$ & 80'000.- = $e_{22}$ \\ \hline
	\end{tabular}
\end{example}
\textbf{Entscheidung wegen rationalem Gegenspieler:} Umweltzustände werden durch einen rationalen Gegenspieler bestimmt (und nicht durch Zufall)., gehört zur Spieltheorie.
\begin{example}
	Angenommen zwei verschiedene Ölkonzerne würden jeweils überlegen, ob sie am Stadtrand oder im Stadtzentrum eine neue Tankstelle errichten. \\
	Ungünstig wäre sicherlich am gleichen Standort wie der Konkurrent zu bauen. Entsprechend wäre es wünschenswert zu wissen wo der
	Konkurrent baut. Am besten wäre glaubhaft machen zu können, dass man am lukrativsten Standort plant zu bauen. Dann würde der Konkurrent auf den weniger lukrativen ausweichen.
\end{example}

\subsection{Dominanz}
\begin{compactitem}
	\item Alternative ist dominant, wenn sie einer anderen Alternative auf jeden Fall vorzuziehen ist.
	\item Dominiert eine Alternative alle anderen, so ist diese zu präferieren.
	\item Häufig kann so eine Vorauswahl getroffen werden, indem alle dominierten Alternativen entfernt werden.
\end{compactitem}

\subsubsection{Absolute Dominanz}
Der schlechteste Ergebniswert der dominierenden Alternative ist besser als der beste Ergebniswert der
dominierten Alternative.
\begin{example} \\
	\begin{tabular}{|l|l|l|}
		\hline
		Alternative & $z_1$ (keine Umgehungsstrasse) & $z_2$ (Umgehungsstrasse) \\ \hline
		$a_1$ (Stadtzentrum) & 125'000.- & 90'000.- \\ \hline
		$a_2$ (Stadtrand) & 100'000.- & 80'000.- \\ \hline
	\end{tabular}\\ \ \\
	Keine absolute Dominanz, da der schlechteste Wert der Alternative $a_1$ schlechter als der beste der Alternative $a_2$ ist.
\end{example}

\subsubsection{Zustandsdominanz}
Eine Alternative ist in jedem Zustand (Umweltzustand) besser als (bzw. gleich gut wie) die andere Alternative.
\begin{example} 
	In Tabelle von \aszeichen{Absolute Dominanz} ist unabhängig vom Umweltzustand die Alternative $a_1$ immer besser. $a_1$ dominiert $a_2$.
\end{example}

\subsection{Ziele}
\textbf{Komplementäre Ziele:} Durch die Verfolgung eines Ziels wird auch das andere Ziel optimal erreicht. \\
\textbf{Neutrale Ziele:} Die Verfolgung eines Ziels beeinflusst das andere Ziel nicht. \\
\textbf{Konkurrierende Ziele:} Die Realisierung des einen Ziels beeinträchtigt die Erreichung des anderen Ziels.

\subsubsection{Effizienz}
Eine Alternative ist effizient, wenn es keine andere Alternative gibt, die bezüglich mindestens eines Zieles besser und bezüglich keines Ziels schlechter ist.
\begin{example} 	
	Ein Chemieunternehmen will sowohl den Absatz als auch den Gewinn maximieren. Es vermutet allerdings, dass die zukünftigen Gewinne durch die von der Bevölkerung wahrgenommene Umweltschädigung verursacht durch das Unternehmen beeinflusst werden. Entsprechend gilt es auch die Schäden zu minimieren. Das Unternehmen kann über den Preis den Absatz steuern. Gewinn und Schäden ergeben sich dann automatisch:\\
	\begin{tabular}{|l|l|l|l|l|}
		\hline
		\textbf{Alternative} & \textbf{Preis} & \textbf{Absatz} & \textbf{Gewinn} & \textbf{Umweltschäden} \\ \hline
		$a_1$ & 15 & 800 & 7000 & -4 \\ \hline
		$a_2$ & 20 & 600 & 7000 & -2 \\ \hline
		$a_3$ & 25 & 400 & 6000 & 0 \\ \hline
		$a_4$ & 30 & 200 & 4000 & 0 \\ \hline
	\end{tabular} \\ \ \\
	$a_4$ ist ineffizient, da diese nie besser ist als $a_3$, aber bezüglich Absatz und Gewinn schlechter als $a_3$.
\end{example}
Sinnvoll ist es, sich nur auf die effizienten Alternativen zu beschränken.

\subsubsection{Zieldominanz}
Wenn ein Ziel als das entscheidende erachtet wird, kann die Erfüllung aller anderen Ziele vernachlässigt werden.

\subsubsection{Lexikographische Ordnung}
Bei Verfahren der Lexikographischen Ordnung werden die Ziele sortiert und nur wenn die Ziele identisch sind, wird das nächste Ziel zur Sortierung hinzugezogen.
\begin{example}
	Es wird angenommen, dass für die Chemiefirma die Gewinnmaximierung das Hauptziel ist und die Minimierung der Umweltschäden das zweitwichtigste Ziel. \\
	\begin{tabular}{|l|l|l|l|l|}
		\hline
		\textbf{Alternative} & \textbf{Preis} & \textbf{Absatz} & \textbf{Gewinn} & \textbf{Umweltschäden} \\ \hline
		$a_2$ & 20 & 600 & 7000 & -2 \\ \hline
		$a_1$ & 15 & 800 & 7000 & -4 \\ \hline		
		$a_3$ & 25 & 400 & 6000 & 0 \\ \hline
	\end{tabular}
\end{example}
\textbf{Problem:} Selbst ein minimaler Unterschied beim Hauptziel hat eine grössere Bedeutung als die grösste Abweichung im Nebenziel.

\subsubsection{Nutzenfunktion}
Entscheidungsproblem mit mehreren Zielen kann gelöst werden, indem das Maximum einer Nutzenfunktion, die als Variable die
verschiedenen Ziele enthält, bestimmt wird. \\
Suche den maximalen Wert für $U(e_{i1}, ..., e_{ij})$

\subsubsection{Zielgewichtung}
Spezieller Fall der Nutzenfunktion, bei der die einzelnen Ziele linear in die Nutzenfunktion eingehen.\\
\textbf{Vorgehen:} 
\begin{compactenum}
	\item Gewichte den einzelnen Zielen zuordnen.
	\item Normieren der Werte und anschliessend gewichten.
	\item Ergebniswerte für alle Alternativen addieren.
	\item Gewählt wird Alternative mit höchstem Wert.
\end{compactenum}
\begin{example}
	Eine Firma möchte sowohl den Umsatz	wie auch den Gewinn maximieren. \\
	\begin{tabular}{|l|l|l|}
		\hline
		\textbf{Alternative} & \textbf{Umsatz} & \textbf{Gewinn} \\ \hline
		$a_1$ & 800'000 & 7000 \\ \hline
		$a_2$ & 600'000 & 8000 \\ \hline		
	\end{tabular} \\ \ \\
	Angenommen der Firma wäre nun die Gewinnmaximierung dreimal wichtiger als die Umsatzmaximierung würden die Gewichtungsfaktoren folgendermassen lauten: 
	\begin{compactitem}
		\item $a_1$ : 0.25 * 800'000 + 0.75 * 7000 = 205'250
		\item $a_2$ : 0.25 * 600'000 + 0.75 * 8000 = 156'000
	\end{compactitem}
\end{example}

\subsubsection{Normierung}
Eine vorgängige Normierung der Ergebnisse kann mittels folgender Formel erfolgen: \\
$\text{normierter Wert} = \frac{\text{zu normierender Wert} - \text{minimaler Wert}}{\text{maximaler Wert} - \text{minimaler Wert}}$
\begin{example}
	\begin{compactitem}
		\item $a_1$ : 0.25 * 1 + 0.75 * 1 = 0.25
		\item $a_2$ : 0.25 * 0 + 0.75 * 0 = 0.75
	\end{compactitem}
\end{example}

\subsection{Entscheidung unter Ungewissheit}
Situation mit mehreren Umweltzuständen, für die die Eintrittswahrscheinlichkeiten nicht bekannt sind.

\subsubsection{Maximin-Regel}
\textbf{Regel:} Man wähle die Handlungsalternative mit dem maximale Minimum. \\
\textbf{Intention:} Extrem risikoscheue Entscheidungsregel, da nur das schlechteste mögliche Ergebnis in die Beurteilung einfliesst.\\
\textbf{Kritik:} Realitätsfremd und es wird nur ein einziger Wert einer Alternative berücksichtigt.
\begin{example} \\
	\begin{tabular}{|l|l|l|l|l|l|}
		\hline
		& $z_1$ & $z_2$ & $z_3$ & $z_4$ & Minimum \\ \hline
		$a_1$ & 60 & 30 & 50 & 60 & \textbf{30} \\ \hline
		$a_2$ & 10 & 10 & 10 & 140 & 10 \\ \hline
		$a_3$ & -30 & 100 & 120 & 130 & -30 \\ \hline		
	\end{tabular}
\end{example}

\subsubsection{Maximax-Regel}
\textbf{Regel:} Man wähle die Handlungsalternative mit dem maximale Maximum. \\
\textbf{Intention:} Extrem risikofreudige (optimistische) Entscheidungsregel, da nur das bestmögliche Ergebnis in die Beurteilung einfliesst.\\
\textbf{Kritik:} Extrem realitätsfremd und es wird ebenfalls nur ein einziger Wert einer Alternative berücksichtigt.
\begin{example} \\
	\begin{tabular}{|l|l|l|l|l|l|}
		\hline
		& $z_1$ & $z_2$ & $z_3$ & $z_4$ & Maximum \\ \hline
		$a_1$ & 60 & 30 & 50 & 60 & 60 \\ \hline
		$a_2$ & 10 & 10 & 10 & 140 & \textbf{140} \\ \hline
		$a_3$ & -30 & 100 & 120 & 130 & 130 \\ \hline		
	\end{tabular}
\end{example}

\subsubsection{Hurwicz-Regel}
\textbf{Regel:} Kombination aus der Maximin- und der Maximax-Regel durch Einführung eines Optimismusparameters $\lambda$. \\
$\Phi(a_i) = \lambda * \max_j(e_{ij}) + (1 - \lambda) * \min_j(e_{ij})$ mit $0 \leq \lambda \leq 1$ \\
\textbf{Intention:} Entscheidungsregel für Entscheider, die weder absolut optimistisch noch absolut pessimistisch sind.\\
\textbf{Kritik:} In vielen Fällen realitätsfremd und es werden nur zwei Werte jeder Alternative berücksichtigt.
\begin{example} 
	$\lambda$ = 0.4 \\
	\begin{tabular}{|l|l|l|l|l|l|}
		\hline
		& $z_1$ & $z_2$ & $z_3$ & $z_4$ & $\Phi(a_i)$ \\ \hline
		$a_1$ & 60 & 30 & 50 & 60 & 0.4 * 60 + 0.6 * 30 = 42 \\ \hline
		$a_2$ & 10 & 10 & 10 & 140 & 0.4 * 140 + 0.6 * 10 = \textbf{62} \\ \hline
		$a_3$ & -30 & 100 & 120 & 130 & 0.4 * 130 + 0.6 * (-30) = 34 \\ \hline		
	\end{tabular}
\end{example}

\subsubsection{Savage-Niehans-Regel}
\textbf{Regel:} \aszeichen{Regel des kleinsten Bedauerns} \\
\textbf{Intention:} Die Regel versucht das maximal mögliche Bedauern zu minimieren.\\
\textbf{Vorgehen:} Aufstellen einer Matrix des Bedauerns, indem die Maxima der Umweltzustände ermittelt und dann pro Handlungsoption die maximal mögliche Differenz zu diesem Wert berechnet wird. Das maximal mögliche Bedauern, ist dann das Zeilenmaximum. \\
\textbf{Kritik:} Bezüglich der Bedauernsmatrix beruht die Entscheidung auf den Prinzipien des Minimax-Kriteriums. Daher werden auch bei diesem Kriterium nicht alle Informationen verwendet, und es drückt letztlich eine pessimistische Haltung aus. 
\begin{example} \\
	\begin{tabular}{|l|l|l|l|l||l|l|l|l|l|}
		\hline
		& $z_1$ & $z_2$ & $z_3$ & $z_4$ & $z_1$ & $z_2$ & $z_3$ & $z_4$ & Maximum\\ \hline
		$a_1$ & \textbf{60} & 30 & 50 & 60 & 60 - 60 = 0 & 100 - 30 = 70 & 120 - 50 = 70 & 140 - 60 = 80 & \textbf{80} \\ \hline
		$a_2$ & 10 & 10 & 10 & \textbf{140} & 60 - 10 = 50 & 100 - 10 = 90 & 120 - 10 = 110 & 140 - 140 = 0 & 110 \\ \hline
		$a_3$ & -30 & \textbf{100} & \textbf{120} & 130 & 60 - -30 = 90 & 100 - 100 = 0 & 120 - 120 = 0 & 140 - 130 = 10 & 90 \\ \hline		
	\end{tabular}
\end{example}

\subsubsection{Laplace-Kriterium}
\textbf{Regel:} Beim Laplace-Kriterium wird für jede Alternative der Durchschnitt aller Umweltzustände gebildet. \\
\textbf{Intention:} \aszeichen{Prinzip des mangelnden Grundes} d.h. da es keinen Grund gibt, dass ein Umweltzustand wahrscheinlicher ist als ein anderer wird davon ausgegangen, dass alle Zustände gleich wahrscheinlich sind.\\
\textbf{Kritik:} Es lässt sich festhalten, dass wahrscheinlich die meisten Alternative $a_3$ gewählt hätten und somit das Laplace-Kriterium verglichen mit den anderen Regeln realitätsnäher ist. Die Entscheidung wird beeinflusst durch das Hinzufügen praktisch identischer Spalten, d.h. für tatsächlich gleichwahrscheinliche Umweltzustände. Das Kriterium ist nur für risikoneutrale Entscheidungsträger rational, da Ausreisser gleichgewichtet werden.
\begin{example} \\
	\begin{tabular}{|l|l|l|l|l|l|}
		\hline
		& $z_1$ & $z_2$ & $z_3$ & $z_4$ & $\Phi(a_i)$ \\ \hline
		$a_1$ & 60 & 30 & 50 & 60 & (60 + 30 + 50 + 60) / 4 = 50 \\ \hline
		$a_2$ & 10 & 10 & 10 & 140 & (10 + 10 + 10 + 140) / 4 = 42.5 \\ \hline
		$a_3$ & -30 & 100 & 120 & 130 & (-30 + 100 + 120 + 130) / 4 = \textbf{80} \\ \hline		
	\end{tabular}
\end{example}

\subsection{Entscheidung unter Risiko}
\subsubsection{$\mu$-Kriterium/Bayes-Regel}
\textbf{Erwartungswert:} Der Erwartungswert ist der Wert, der sich als Mittelwert ergibt, wenn man die Situation unendlich oft wiederholen würde. Der Erwartungswert einer Alternative berechnet man, indem die Ergebniswerte mit den der dazugehörigen Wahrscheinlichkeit multipliziert und die sich so ergebenden Werte addiert werden. \\
\textbf{risikoscheu:} Der Entscheidungsträger bevorzugt einen niedrigeren Erwartungswert, wenn dieser mehr Sicherheit bringt.\\
\textbf{risikoneutral:} Dem Entscheidungsträger ist die Schwankung der Ergebnisse egal, er orientiert sich nur nach dem Erwartungswert.\\
\textbf{risikofreudig:} Der Entscheidungsträger verzichtet auf einen höheren Erwartungswert zugunsten einer grösseren Streuung der Ergebnisse.\\
\textbf{Fazit:} Für risikoneutrale Entscheidungsträger ist eine Entscheidung nach dem $\mu$-Kriterium rational. Darüber hinaus liefert das $\mu$-Kriterium bei oft wiederholten Entscheidungen bzw. Entscheidungen, deren Tragweite in Relation zum eigenen Vermögen nur klein ist, brauchbare Ergebnisse.
\begin{example} \\
	\begin{tabular}{|l|l|l|l|l|}
		\hline
		& $z_1$ & $z_2$ & $z_3$ & $\mu$ (Erwartungswert) \\
		& $p_1$ = 0.5 & $p_2$ = 0.2 & $p_3$ = 0.3 & \\ \hline
		$a_1$ & 30 & 20 & 20 & 0.5 * 30 + 0.2 * 20  + 0.3 * 20 = 25 \\ \hline
		$a_2$ & 140 & -40 & -30 & 0.5 * 140 + 0.2 * (-40) + 0.3 * (-30) = \textbf{53} \\ \hline
		$a_3$ & 40 & 10 & 60 & 0.5 * 40 + 0.2 * 10 + 0.3 * 60 = 40 \\ \hline		
	\end{tabular}
\end{example}

\subsubsection{Entscheidungsbaum}
Entscheidungsbäume sind geordnete, gerichtete Bäume, die der Darstellung von Entscheidungsregeln dienen. Die grafische Darstellung
als Baumdiagramm veranschaulicht hierarchisch aufeinanderfolgende Entscheidungen. \\
Ein Entscheidungsbaum besteht immer aus ...
\begin{compactitem}
	\item ... einem Wurzelknoten
	\item ... beliebig vielen inneren Knoten (logische Regel)
	\item ... mindestens zwei Blättern (Antwort auf Entscheidungsproblem)
\end{compactitem}
\textbf{Darstellung:}
\begin{compactenum}
	\item Die Zeit in einer Entscheidungsbaum-Darstellung läuft von	links nach rechts.
	\item Ereignisknoten werden als Kreise dargestellt,	Entscheidungsknoten als Quadrate.
	\item Die Kanten aus einem Entscheidungsknoten werden mit den	möglichen Entscheidungen angeschrieben. Alle möglichen Entscheidungen müssen vorkommen.
	\item  Die Kanten aus einem Ereignisknoten werden mit den möglichen	Ereignissen angeschrieben. Alle möglichen Ereignisse müssen
	vorkommen.
\end{compactenum}
\textbf{Entscheiden:}
\begin{compactenum}
	\item Entscheidungsbaum quantifizieren:
	\begin{compactitem}
		\item Bei jedem Ereignisknoten: Wahrscheinlichkeiten definieren
		\item An jedem freien Kantenende: CHF anschreiben
		\item Ein Ereignisknoten erhält als \aszeichen{expected monetary value} (EMV) das wahrscheinlichkeits-gewichtete Mittel der EMVs der unmittelbar rechts liegenden Knoten.
		\item Ein Entscheidungsknoten erbt den grössten EMV der unmittelbar rechts liegenden Knoten.
	\end{compactitem}
	\item Im Entscheidungsknoten wird jene Kante gewählt, die den höchsten Erwartungswert aufweist.
\end{compactenum}
\begin{example}
	Soll ein Angebot von CHF 10'000 angenommen werden, wenn bei seiner Ablehnung ein nächstes (letztes) Angebot auftauchen wird, das
	\begin{compactitem}
	\item mit 60\% Wahrscheinlichkeit CHF 15'000
	\item mit 30\% Wahrscheinlichkeit CHF 8'000
	\item und mit 10\% nichts wert ist?
	\end{compactitem}
	\begin{tikzpicture}[node distance = 0.6cm and 2.5cm]
		\node (resultA) [answer] {A: CHF 10'000};
		\node (resultB) [answer, below=of resultA] {B: CHF 15'000};
		\node (resultC) [answer, below=of resultB] {C: CHF 8'000};
		\node (resultD) [answer, below=of resultC] {D: CHF 0};
		\node (waitForB) [event, left=of resultC, label=north:{CHF 11'400}] {Alternative B};
		\node (J) [decision, left=of waitForB] {J};
		\draw (waitForB) -- node[anchor=south]{0.6} (resultB.west);
		\draw (waitForB) -- node[anchor=south]{0.3} (resultC);
		\draw (waitForB) -- node[anchor=south]{0.1} (resultD.west);
		\draw (J) -- node[anchor=south]{Warten auf B} (waitForB);
		\draw (J) |- node[anchor=north west]{A annehmen} (resultA);
	\end{tikzpicture}\\
	\aszeichen{Warten auf B} hat mit CHF 11'400 einen höheren Erwartungswert (EMV) als A.
\end{example}

\subsubsection{Bernoulli-Prinzip}
\textbf{Idee:} Die Grundidee besteht darin, den Erwartungswert nicht von den Ergebniswerten sondern von den Nutzenwerten zu berechnen und diesen dann als Präferenzwert zu betrachten. Zur Bestimmung der Nutzenfunktion kann die Bernoulli-Befragung
dienen, die eine Risiko-Nutzen-Funktion (RNF) oder kurz Nutzenfunktion $u(e_{ij})$ herausgibt.
\begin{example}
	Folgende Nutzenfunktion sei gegeben: $u(e_{ij}) = 300 * e_{ij} - e_{ij}^2$ \\
	\begin{tabular}{|l|l|l|l||l|l|l||l|}
		\hline
		& $z_1$ & $z_2$ & $z_3$ & $u(e_{i1})$ & $u(e_{i2})$ & $u(e_{i3})$ & $E(u(e_{ij}))$ \\
		& $p_1$ = 0.5 & $p_2$ = 0.2 & $p_3$ = 0.3 & $p_1$ = 0.5 & $p_2$ = 0.2 & $p_3$ = 0.3 & \\ \hline
		$a_1$ & 30 & 20 & 20 & 8'100 & 5'600 & 5'600 & 0.5*8'100+0.2*5'600+0.3*5'600=6'850\\ \hline
		$a_2$ & 140 & -40 & -30 & 22'400 & -13'600 & -9'600 & 0.5*22'400+0.2*(-13'600)+0.3*(-9'900)=5'510\\ \hline
		$a_3$ & 40 & 10 & 60 & 10'400 & 2'900 & 14'400 & 0.5*10'400+0.2*2'900+0.3*14'400=\textbf{10'100}\\ \hline
	\end{tabular} \\ \ \\
	Alternative 3 hat den höchsten Erwartungsnutzen. Da der Entscheider auf den hohen Erwartungswert der Alternative 2 ($\mu$  = 53) zugunsten der Alternative 3 ($\mu$ = 40) verzichtet, weil die Ergebnisse bei der Variante 2 unsicherer sind, ist er risikoscheu.
\end{example}
Wähle die Alternative, für die sich der grösste Erwartungsnutzen ergibt.