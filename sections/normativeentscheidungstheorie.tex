\section{Normative Entscheidungstheorie}
\begin{example}[Beispiel für Kapitel 3.1 - 3.4]
	Angenommen ein Mineralölkonzern strebe nach Gewinnmaximierung und überprüfe die Eröffnung einer neuen Tankstelle. In Frage kommen zwei Standorte:
	\begin{multicols}{2}
		\begin{compactenum}
			\item Standort im Stadtzentrum (a1): 
			\begin{compactitem}
				\item Erwarteter Gewinn 125'000.- CHF 
				\item Erwarteter Umsatz 2'000'000.- CHF
			\end{compactitem}
			\item Standort am Stadtrand (a2): 
			\begin{compactitem}
				\item Erwarteter Gewinn 150'000.- CHF 
				\item Erwarteter Umsatz 1'800'000.- CHF
			\end{compactitem}
		\end{compactenum}
	\end{multicols}
\end{example}

\subsection{Entscheidung unter Unsicherheit}
\textbf{Hauptsächliche Beschäftigungsgebiet:} Unsicherheit und nicht der Zielkonflikt \\
\textbf{Umweltzustände:} Faktoren auf welche man selber aber keinen Einfluss hat, die aber die Entscheidung beeinflussen\\
Bei Entscheidungssituationen in denen eine bestimmte Entscheidung nicht mit einem einzigen möglichen Ereignis verbunden ist, wird von einer Entscheidung unter Unsicherheit gesprochen.

\subsection{Ergebnismatrix}
Darstellungsform für Entscheidungen unter Unsicherheit wobei die Handlungsalternativen ($a_i$) in den Zeilen und die verschiedenen Umweltzustände ($z_i$) in den Spalten dargestellt werden. 
\begin{example}
	Dem Mineralölkonzern ist bekannt, dass eine weiträumige Umgehungsstrasse geplant ist, die aber höchst umstritten ist.
	\begin{compactitem}
		\item Diese würden den Verkehr im Stadtzentrum nicht tangieren und auf CHF 125'000.- belassen.
		\item Am Stadtrand würde sich aber eine drastische Veränderung ergeben,	die den erwarteten Gewinn auf CHF 80'000.- reduzieren liesse.
	\end{compactitem}
	Welcher Standort wäre unter der Prämisse der Gewinnmaximierung nun zu präferieren? \\
	\begin{tabular}{|l|l|l|}
		\hline
		& $z_1$ (keine Umgehungsstrasse) & $z_2$ (Umgehungsstrasse) \\ \hline
		$a_1$ (Stadtzentrum) & 125'000.- = $e_{11}$ & 125'000.- = $e_{11}$ \\ \hline
		$a_2$ (Stadtrand) & 150'000.- = $e_{21}$ & 80'000.- = $e_{22}$ \\ \hline
	\end{tabular}
\end{example}

\subsubsection{Zustandsraum}
\begin{compactitem}
	\item Raum aller möglicher Entscheidungsergebnisse.
	\item Die Entscheidungsmatrix beschreibt die Entscheidungssituation nur dann hinreichend, wenn der Zustandsraum voll-ständig erfasst wird.
	\item In der Realität oft nicht gegeben.
\end{compactitem}

\subsection{Klassifikation}
\textbf{Entscheidung unter Risiko:} Falls den Umweltzuständen Eintrittswahrscheinlichkeiten zugeordnet werden können. \\
\textbf{Entscheidung unter Ungewissheit:} Falls den Umweltzuständen keine Wahrscheinlichkeiten zugeordnet werden können. \\
Wahrscheinlichkeiten müssen nicht mathematisch fundiert sein (objektiv), sondern können sich auch aus Sicht des Entscheiders ergeben (subjektiv).
\begin{example}
	Angenommen der Ölkonzern schätzt die Wahrscheinlichkeit für den Bau	der Umgehungsstrasse auf 30\%. \\
	\begin{tabular}{|l|l|l|}
		\hline
		Alternative & $z_1$ (keine Umgehungsstrasse) & $z_2$ (Umgehungsstrasse) \\ 
		& $p_1$ = 0.7 & $p_2$ = 0.3 \\ \hline
		$a_1$ (Stadtzentrum) & 125'000.- = $e_{11}$ & 125'000.- = $e_{11}$ \\ \hline
		$a_2$ (Stadtrand) & 150'000.- = $e_{21}$ & 80'000.- = $e_{22}$ \\ \hline
	\end{tabular}
\end{example}
\textbf{Entscheidung wegen rationalem Gegenspieler:} Umweltzustände werden durch einen rationalen Gegenspieler bestimmt (und nicht durch Zufall)., gehört zur Spieltheorie.
\begin{example}
	Angenommen zwei verschiedene Ölkonzerne würden jeweils überlegen, ob sie am Stadtrand oder im Stadtzentrum eine neue Tankstelle errichten. \\
	Ungünstig wäre sicherlich am gleichen Standort wie der Konkurrent zu bauen. Entsprechend wäre es wünschenswert zu wissen wo der
	Konkurrent baut. Am besten wäre glaubhaft machen zu können, dass man am lukrativsten Standort plant zu bauen. Dann würde der Konkurrent auf den weniger lukrativen ausweichen.
\end{example}

\subsection{Dominanz}
\begin{compactitem}
	\item Alternative ist dominant, wenn sie einer anderen Alternative auf jeden Fall vorzuziehen ist.
	\item Dominiert eine Alternative alle anderen, so ist diese zu präferieren.
	\item Häufig kann so eine Vorauswahl getroffen werden, indem alle dominierten Alternativen entfernt werden.
\end{compactitem}

\subsubsection{Absolute Dominanz}
Der schlechteste Ergebniswert der dominierenden Alternative ist besser als der beste Ergebniswert der
dominierten Alternative.
\begin{example} \\
	\begin{tabular}{|l|l|l|}
		\hline
		Alternative & $z_1$ (keine Umgehungsstrasse) & $z_2$ (Umgehungsstrasse) \\ \hline
		$a_1$ (Stadtzentrum) & 125'000.- & 90'000.- \\ \hline
		$a_2$ (Stadtrand) & 100'000.- & 80'000.- \\ \hline
	\end{tabular}\\ \ \\
	Keine absolute Dominanz, da der schlechteste Wert der Alternative $a_1$ schlechter als der beste der Alternative $a_2$ ist.
\end{example}

\subsubsection{Zustandsdominanz}
Eine Alternative ist in jedem Zustand (Umweltzustand) besser als (bzw. gleich gut wie) die andere Alternative.
\begin{example} 
	In Tabelle von \aszeichen{Absolute Dominanz} ist unabhängig vom Umweltzustand die Alternative $a_1$ immer besser. $a_1$ dominiert $a_2$.
\end{example}

\subsection{Ziele}
\textbf{Komplementäre Ziele:} Durch die Verfolgung eines Ziels wird auch das andere Ziel optimal erreicht. \\
\textbf{Neutrale Ziele:} Die Verfolgung eines Ziels beeinflusst das andere Ziel nicht. \\
\textbf{Konkurrierende Ziele:} Die Realisierung des einen Ziels beeinträchtigt die Erreichung des anderen Ziels.

\subsubsection{Effizienz}
Eine Alternative ist effizient, wenn es keine andere Alternative gibt, die bezüglich mindestens eines Zieles besser und bezüglich keines Ziels schlechter ist.
\begin{example} 	
	Ein Chemieunternehmen will sowohl den Absatz als auch den Gewinn maximieren. Es vermutet allerdings, dass die zukünftigen Gewinne durch die von der Bevölkerung wahrgenommene Umweltschädigung verursacht durch das Unternehmen beeinflusst werden. Entsprechend gilt es auch die Schäden zu minimieren. Das Unternehmen kann über den Preis den Absatz steuern. Gewinn und Schäden ergeben sich dann automatisch:\\
	\begin{tabular}{|l|l|l|l|l|}
		\hline
		\textbf{Alternative} & \textbf{Preis} & \textbf{Absatz} & \textbf{Gewinn} & \textbf{Umweltschäden} \\ \hline
		$a_1$ & 15 & 800 & 7000 & -4 \\ \hline
		$a_2$ & 20 & 600 & 7000 & -2 \\ \hline
		$a_3$ & 25 & 400 & 6000 & 0 \\ \hline
		$a_4$ & 30 & 200 & 4000 & 0 \\ \hline
	\end{tabular} \\ \ \\
	$a_4$ ist ineffizient, da diese nie besser ist als $a_3$, aber bezüglich Absatz und Gewinn schlechter als $a_3$.
\end{example}
Sinnvoll ist es, sich nur auf die effizienten Alternativen zu beschränken.

\subsubsection{Zieldominanz}
Wenn ein Ziel als das entscheidende erachtet wird, kann die Erfüllung aller anderen Ziele vernachlässigt werden.

\subsubsection{Lexikographische Ordnung}
Bei Verfahren der Lexikographischen Ordnung werden die Ziele sortiert und nur wenn die Ziele identisch sind, wird das nächste Ziel zur Sortierung hinzugezogen.
\begin{example}
	Es wird angenommen, dass für die Chemiefirma die Gewinnmaximierung das Hauptziel ist und die Minimierung der Umweltschäden das zweitwichtigste Ziel. \\
	\begin{tabular}{|l|l|l|l|l|}
		\hline
		\textbf{Alternative} & \textbf{Preis} & \textbf{Absatz} & \textbf{Gewinn} & \textbf{Umweltschäden} \\ \hline
		$a_2$ & 20 & 600 & 7000 & -2 \\ \hline
		$a_1$ & 15 & 800 & 7000 & -4 \\ \hline		
		$a_3$ & 25 & 400 & 6000 & 0 \\ \hline
	\end{tabular}
\end{example}
\textbf{Problem:} Selbst ein minimaler Unterschied beim Hauptziel hat eine grössere Bedeutung als die grösste Abweichung im Nebenziel.

\subsubsection{Nutzenfunktion}
Entscheidungsproblem mit mehreren Zielen kann gelöst werden, indem das Maximum einer Nutzenfunktion, die als Variable die
verschiedenen Ziele enthält, bestimmt wird. \\
Suche den maximalen Wert für $U(e_{i1}, ..., e_{ij})$

\subsubsection{Zielgewichtung}
Spezieller Fall der Nutzenfunktion, bei der die einzelnen Ziele linear in die Nutzenfunktion eingehen.
\textbf{Vorgehen:} 
\begin{compactenum}
	\item Gewichte den einzelnen Zielen zuordnen.
	\item Normieren der Werte und anschliessend gewichten.
	\item Ergebniswerte für alle Alternativen addieren.
	\item Gewählt wird Alternative mit höchstem Wert.
\end{compactenum}
\begin{example}
	Eine Firma möchte sowohl den Umsatz	wie auch den Gewinn maximieren. \\
	\begin{tabular}{|l|l|l|}
		\hline
		\textbf{Alternative} & \textbf{Umsatz} & \textbf{Gewinn} \\ \hline
		$a_1$ & 800'000 & 7000 \\ \hline
		$a_2$ & 600'000 & 8000 \\ \hline		
	\end{tabular} \\ \ \\
	Angenommen der Firma wäre nun die Gewinnmaximierung dreimal wichtiger als die Umsatzmaximierung würden die Gewichtungsfaktoren folgendermassen lauten: 
	\begin{compactitem}
		\item $a_1$ : 0.25 * 800'000 + 0.75 * 7000 = 205'250
		\item $a_2$ : 0.25 * 600'000 + 0.75 * 8000 = 156'000
	\end{compactitem}
\end{example}

\subsubsection{Normierung}
Eine vorgängige Normierung der Ergebnisse kann mittels folgender Formel erfolgen: \\
$\text{normierter Wert} = \frac{\text{zu normierender Wert} - \text{minimaler Wert}}{\text{maximaler Wert} - \text{minimaler Wert}}$
\begin{example}
	\begin{compactitem}
		\item $a_1$ : 0.25 * 1 + 0.75 * 1 = 0.25
		\item $a_2$ : 0.25 * 0 + 0.75 * 0 = 0.75
	\end{compactitem}
\end{example}