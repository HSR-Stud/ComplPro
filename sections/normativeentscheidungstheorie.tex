\section{Normative Entscheidungstheorie}
\begin{example}[Beispiel für dieses Kapitel]
	Angenommen ein Mineralölkonzern strebe nach Gewinnmaximierung und überprüfe die Eröffnung einer neuen Tankstelle. In Frage kommen zwei Standorte:
	\begin{multicols}{2}
		\begin{compactenum}
			\item Standort im Stadtzentrum (a1): 
			\begin{compactitem}
				\item Erwarteter Gewinn 125'000.- CHF 
				\item Erwarteter Umsatz 2'000'000.- CHF
			\end{compactitem}
			\item Standort am Stadtrand (a2): 
			\begin{compactitem}
				\item Erwarteter Gewinn 150'000.- CHF 
				\item Erwarteter Umsatz 1'800'000.- CHF
			\end{compactitem}
		\end{compactenum}
	\end{multicols}
\end{example}

\subsection{Entscheidung unter Unsicherheit}
\textbf{Hauptsächliche Beschäftigungsgebiet:} Unsicherheit und nicht der Zielkonflikt \\
\textbf{Umweltzustände:} Faktoren auf welche man selber aber keinen Einfluss hat, die aber die Entscheidung beeinflussen\\
Bei Entscheidungssituationen in denen eine bestimmte Entscheidung nicht mit einem einzigen möglichen Ereignis verbunden ist, wird von einer Entscheidung unter Unsicherheit gesprochen.

\subsection{Ergebnismatrix}
Darstellungsform für Entscheidungen unter Unsicherheit wobei die Handlungsalternativen ($a_i$) in den Zeilen und die verschiedenen Umweltzustände ($z_i$) in den Spalten dargestellt werden. 
\begin{example}
	Dem Mineralölkonzern ist bekannt, dass eine weiträumige Umgehungsstrasse geplant ist, die aber höchst umstritten ist.
	\begin{compactitem}
		\item Diese würden den Verkehr im Stadtzentrum nicht tangieren und auf CHF 125'000.- belassen.
		\item Am Stadtrand würde sich aber eine drastische Veränderung ergeben,	die den erwarteten Gewinn auf CHF 80'000.- reduzieren liesse.
	\end{compactitem}
	Welcher Standort wäre unter der Prämisse der Gewinnmaximierung nun zu präferieren? \\
	\begin{tabular}{|l|l|l|}
		\hline
		& $z_1$ (keine Umgehungsstrasse) & $z_2$ (Umgehungsstrasse) \\ \hline
		$a_1$ (Stadtzentrum) & 125'000.- = $e_{11}$ & 125'000.- = $e_{11}$ \\ \hline
		$a_2$ (Stadtrand) & 150'000.- = $e_{21}$ & 80'000.- = $e_{22}$ \\ \hline
	\end{tabular}
\end{example}

\subsubsection{Zustandsraum}
\begin{compactitem}
	\item Raum aller möglicher Entscheidungsergebnisse.
	\item Die Entscheidungsmatrix beschreibt die Entscheidungssituation nur dann hinreichend, wenn der Zustandsraum voll-ständig erfasst wird.
	\item In der Realität oft nicht gegeben.
\end{compactitem}

\subsection{Klassifikation}
\textbf{Entscheidung unter Risiko:} Falls den Umweltzuständen Eintrittswahrscheinlichkeiten zugeordnet werden können. \\
\textbf{Entscheidung unter Ungewissheit:} Falls den Umweltzuständen keine Wahrscheinlichkeiten zugeordnet werden können. \\
Wahrscheinlichkeiten müssen nicht mathematisch fundiert sein (objektiv), sondern können sich auch aus Sicht des Entscheiders ergeben (subjektiv).
\begin{example}
	Angenommen der Ölkonzern schätzt die Wahrscheinlichkeit für den Bau	der Umgehungsstrasse auf 30\%. \\
	\begin{tabular}{|l|l|l|}
		\hline
		& $z_1$ (keine Umgehungsstrasse) & $z_2$ (Umgehungsstrasse) \\ 
		& $p_1$ = 0.7 & $p_2$ = 0.3 \\ \hline
		$a_1$ (Stadtzentrum) & 125'000.- = $e_{11}$ & 125'000.- = $e_{11}$ \\ \hline
		$a_2$ (Stadtrand) & 150'000.- = $e_{21}$ & 80'000.- = $e_{22}$ \\ \hline
	\end{tabular}
\end{example}
\textbf{Entscheidung wegen rationalem Gegenspieler:} Umweltzustände werden durch einen rationalen Gegenspieler bestimmt (und nicht durch Zufall)., gehört zur Spieltheorie.
\begin{example}
	Angenommen zwei verschiedene Ölkonzerne würden jeweils überlegen, ob sie am Stadtrand oder im Stadtzentrum eine neue Tankstelle errichten. \\
	Ungünstig wäre sicherlich am gleichen Standort wie der Konkurrent zu bauen. Entsprechend wäre es wünschenswert zu wissen wo der
	Konkurrent baut. Am besten wäre glaubhaft machen zu können, dass man am lukrativsten Standort plant zu bauen. Dann würde der Konkurrent auf den weniger lukrativen ausweichen.
\end{example}

\subsection{Dominanz}
\begin{compactitem}
	\item Alternative ist dominant, wenn sie einer anderen Alternative auf jeden Fall vorzuziehen ist.
	\item Dominiert eine Alternative alle anderen, so ist diese zu präferieren.
	\item Häufig kann so eine Vorauswahl getroffen werden, indem alle dominierten Alternativen entfernt werden.
\end{compactitem}

\subsubsection{Absolute Dominanz}
Der schlechteste Ergebniswert der dominierenden Alternative ist besser als der beste Ergebniswert der
dominierten Alternative.
\begin{example} \\
	\begin{tabular}{|l|l|l|}
		\hline
		& $z_1$ (keine Umgehungsstrasse) & $z_2$ (Umgehungsstrasse) \\ \hline
		$a_1$ (Stadtzentrum) & 125'000.- & 90'000.- \\ \hline
		$a_2$ (Stadtrand) & 100'000.- & 80'000.- \\ \hline
	\end{tabular}\\ \ \\
	Keine absolute Dominanz, da der schlechteste Wert der Alternative $a_1$ schlechter als der beste der Alternative $a_2$ ist.
\end{example}

\subsubsection{Zustandsdominanz}
Eine Alternative ist in jedem Zustand (Umweltzustand) besser als (bzw. gleich gut wie) die andere Alternative.
\begin{example} \\
	\begin{tabular}{|l|l|l|}
		\hline
		& $z_1$ (keine Umgehungsstrasse) & $z_2$ (Umgehungsstrasse) \\ \hline
		$a_1$ (Stadtzentrum) & 125'000.- & 90'000.- \\ \hline
		$a_2$ (Stadtrand) & 100'000.- & 80'000.- \\ \hline
	\end{tabular}\\ \ \\
	In diesem Fall ist unabhängig vom Umweltzustand die Alternative $a_1$ immer besser. $a_1$ dominiert $a_2$.
\end{example}
