\section{Einführung}
\subsection{Komplizierte vs. komplexe Systeme}
Ein System ist komplex/kompliziert, wenn sein Gesamtverhalten trotz vollständiger Information über die Einzelkomponenten schwierig zu verstehen ist – auf Grund ...
\begin{multicols}{2}
	\textbf{Komplex:} ... von Rück- und Nebenwirkungen, Verzögerungen und Nichtlinearitäten der Ursache-Wirkungsbeziehungen. \\ \ \\
	\textbf{Kompliziert:} ... der grossen Zahl von Einzelkomponenten. Ein kompliziertes System ist 'reduzibel', d.h. es kann durch Abstraktion in eine einfaches System überführt werden.
\end{multicols}

\subsection{Fahndungsbild eines komplexen Systemes}
\todo{Allenfalls entfernen, weiss nicht wie wichtig das wirklich ist}
\begin{multicols}{3}
	\begin{compactitem}
		\item Hohe Anzahl Elemente
		\item Nichtlineare Wechselwirkungen zwischen den Elementen
		\item Verzögerte Auswirkungen		
		\item Netzartige Struktur
		\item Negative und positive Rückkoppelungen
		\item Offen
		\item Universell
		\item Dynamisch
		\item Robust
		\item Kreativ und innovativ
		\item Unvorhersehbar
		\item Differenzierte Sensibilität
		\item Nicht kontrollierbar
	\end{compactitem}
\end{multicols}

\subsection{Komplexitätsmanagement}
Komplexitätsmanagement heisst,
\begin{compactitem}
	\item die zentralen Ursache-Wirkungszusammenhänge zu verstehen
	\item zu erkennen, welche Abhängigkeiten nebensächlich sind
	\item steuerbare Einflussgrössen („Stellhebel“) und nicht-steuerbare Einflussgrössen (``exogene Faktoren'') zu erkennen.
	\item was-wäre-wenn-Fragen beantworten zu können (Simulation).
\end{compactitem}
Vereinfachte Abbildungen der Realität oder des Verständnisses über die Realität (Modelle) dienen dabei der Entscheidungsunterstützung.

\subsection{Simulation}
Eine Simulation ist ein virtuelles Experiment. Das virtuelle Experiment soll dabei die gleichen Fragen wie
ein entsprechend reales Experiment beantworten, aber
\begin{multicols}{3}
	\begin{compactitem}
		\item unter kontrollierten Bedingungen („Labor“)
		\item schneller
		\item billiger
		\item ressourcenschonender
		\item ungefährlich
	\end{compactitem}
\end{multicols}